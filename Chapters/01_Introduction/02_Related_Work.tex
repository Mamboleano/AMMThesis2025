\section{Related Work}

\marconote{not sure what I should write here, something regarding the existing formalization and the existing model for sure, then probably how the fee mechanism works in Uniswap v2, differences between that and v3 (maybe future work)}. 

The foundational work \cite{BCL22lmcs} presents an abstract operational model on which the implementation of this thesis is based on. This model does not depend on any specific implementation, but it simply identifies the conditions necessary to ensure desirable properties from AMMs, like net worth preservation and arbitrage conditions. This is the theoretical framework that describes the AMM design that this thesis works with. In addition to \cite{BCL22lmcs}, this thesis delves into the formal verification of AMMs incorporating trading fees, while Bartoletti et al.'s model does not take these fees into account, but proposes this as possible future work. While it is true that all the structural properties presented in \cite{BCL22lmcs} still hold after introducing the trading fee in the model (because they do not depend on it), most of the other properties, especially regarding arbitrage and constant product swap rate function, do not hold or have to be revisited. One key property that is lost after the introduction of the fees into the swap rate is the reversibility, which requires adapting the formal proofs accordingly.

Another foundational work is the Lean 4 implementation of such model provided in \cite{PB24arxiv}. This work models the core functionalities of AMMs, including liquidity provision and token swaps. It also provides machine-checked proofs of the relevant economic properties presented, with the main result being the arbitrage solution, ensuring its mathematical rigor. My work extends this work, adding an implementation for trading fees and providing the appropriate formal definitions in the Lean 4 theorem prover. Unlike \cite{PB24arxiv}, this thesis does not take into account liquidity provision, which means that is solely focuses on token swaps. This is because most of the properties for LPs remain unchanged, while it could be interesting to analyze how their incentives change with the introduction of fees in future work. Like in the former implementation, our implementation provides full machine-checked proofs of all definitions and lemmas provided, with the main result being the arbitrage solutions, which differs from the one presented in \cite{BCL22lmcs} due to the fees mechanisms. While in the previous work this solution coincided with the value that brought the internal swap rate of the AMM to be aligned with the external prices of the AMM tokens, in our model it is not the case.

In summary, my research advances this previous work by incorporating trading fees into the theoretical model and into the implementation, bringing both closer to the real-world scenario and contributing to the development of more secure and robust DeFi systems.  