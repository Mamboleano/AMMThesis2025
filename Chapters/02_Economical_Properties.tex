\chapter{Economic properties of AMMs with trading fees}

In this chapter we study the economic properties of Automated Market Makers (AMMs) that apply a trading fee on swap transactions. Our goal is to analyze how trading fees play a role on the incentives of users and the behavior and properties of swap functions, with keen focus on the constant product swap rate function (~\ref{defi:const-prod}).

All the results of this paper are formalized in Lean 4 and are based on an extension of the model introduced by Bartoletti et al. in \cite{BCL22lmcs}, and the formalization in Lean 4 introduced by Pusceddu et al. in \cite{PB24arxiv}.

To help the reader understand better the importance of each lemma and result, we introduce the economic intuition behind the results, their formal statement, and - when appropriate - an illustrative example. 

\bartnote{conviene organizzare il capitolo in sezioni: provo con delle sezioni un po' arbitrarie, ma poi si possono migliorare}


\section{Gain of swap actions}

One of the key properties of our AMMs model is the user's gain. As we previously stated, we will only focus on the swap transactions; hence, it is only natural to measure the user's gain after a swap and to ask ourselves when this gain is positive. In the following lemma, we define the user's gain in terms of a valid swap transaction.

\albnote{Nel paper LMCS/Lemma 3.2 c'è un caso per il gain di B diverso di A. Perché qui no? Non serve, assumo.}

\marconote{Non serve, su Lean però ho dimostrato anche quel caso, quindi si può aggiungere anche questo teorema}

\citeLean{AMMLib/FeeVersion/Basic.html\#SwapFee.self_gain_eq} % cite-lean(SwapFee.self_gain_eq)
\begin{lemma}{Swap gain}{fee:swap:gain}
  Let $\confG =  \walA{\tokBal[\pmvA]} \mid \amm{\ammR[0]:\tokT[0]}{\ammR[1]:\tokT[1]} \mid \confD$,
  % \bartnote{meglio specificare qui $\mid \walA{\tokBal[\pmvA]}$?}
  and let $\txT = \actAmmSwapExact{\pmvA}{}{x}{\tokT[0]}{\tokT[1]}$
  be enabled in $\confG$.
  Then:
  \begin{align*}
    \gain[{\confG}]{\pmvA}{\txT}
    & = \phantom{-}
      x \cdot \big(
      \SX{x,\ammR[0],\ammR[1]} \, \exchO{\tokT[1]}
      -
      \exchO{\tokT[0]}
      \big)
      \cdot
      \Big(
      1 - \frac{\tokBal[\pmvA]\tokM{\tokT[0]}{\tokT[1]}}{\supply[\confG]{\tokM{\tokT[0]}{\tokT[1]}}}
      \Big)
  \end{align*}
\end{lemma}

This result is not different from the one formalized in \cite{BCL22lmcs}, simply because it does not depend on any particular swap function $\SX{}$. This result also captures how the gain depends on the number of minted tokens held by the user. Since we mostly focus on users that are just swapping tokens (and are not Liquidity Providers), it's worth analyzing the case where $\pmvA$ does not hold any minted token, which will come in hand later on.


\citeLean{AMMLib/FeeVersion/Basic.html\#SwapFee.self_gain_no_mint_eq} % cite-lean(SwapFee.self_gain_eq)
\begin{lemma}{Swap gain no mint}{fee:swap:gain_no_mint}
% \bartnote{questo sembra un caso particolare, e lo metterei fuori dal lemma}
  Let $\confG =  \walA{\tokBal[\pmvA]} \mid \amm{\ammR[0]:\tokT[0]}{\ammR[1]:\tokT[1]} \mid \confD$ be such that $\tokBal[\pmvA]\tokM{\tokT[0]}{\tokT[1]} = 0$,
  and let $\txT = \actAmmSwapExact{\pmvA}{}{x}{\tokT[0]}{\tokT[1]}$
  be enabled in $\confG$.
  Then:

  \begin{align*}
    \gain[{\confG}]{\pmvA}{\txT}
    & = \phantom{-}
      x \cdot \big(
      \SX{x,\ammR[0],\ammR[1]} \, \exchO{\tokT[1]}
      -
      \exchO{\tokT[0]}
      \big)
  \end{align*}
\end{lemma}

With this result, we can finally answer one of the key questions we posed ourselves: when the user's gain is positive upon a valid swap transaction. The following lemma defines this in a more general way, stating that the gain is strictly positive \emph{iff} the swap rate is strictly greater than the external exchange rate.

\albnote{Quando aggiungerai il testo per guidare il lettore, conviene spiegare perché alcuni risultati escludono i  casi dove A non ha minted token.}

\marconote{La ragione è che ci stiamo concentrando su player che non hanno liquidity? Oppure che i risultati si fanno molto più complessi se consideriamo anche i minted tokens? Il lemma di sotto per esempio non vale nel caso in cui il player ha dei minted tokens}

\citeLean{AMMLib/FeeVersion/Basic.html\#SX.fee.swaprate_vs_exchrate} % cite-lean(SX.fee.swaprate_vs_exchrate)

\begin{lemma}{Swap rate \emph{vs.} exchange rate}{fee:swap:gain-SX-X-fee}
  Let $\confG = \walA{\tokBal[\pmvA]} \mid \amm{\ammR[0]:\tokT[0]}{\ammR[1]:\tokT[1]} \mid \confD$ 
  be such that $\tokBal{\tokM{\tokT[0]}{\tokT[1]}} = 0$, and
  let $\txT = \actAmmSwapExact{\pmvA}{}{x}{\tokT[0]}{\tokT[1]}$ 
  be enabled in $\confG$.
  Then: 
  \begin{align*}
      \gain[{\confG}]{\pmvA}{\txT} \circ 0
      \iff
      \SX{x,\ammR[0],\ammR[1]} \circ \X{\tokT[0],\tokT[1]}
       && \text{for $\circ \in \setenum{<, =, >}$}
  \end{align*}
\end{lemma}

This lemma is the first step towards finding the arbitrage solution, since it defines when it is convenient for a user to perform a swap transaction, comparing the swap rate with the external prices of the swapped tokens $\tokT[0]$ and $\tokT[1]$.

\marconote{In the introduction, it is probably convenient to state that the ultimate goal is to define the arbitrage solution, i.e. what is the value that maximizes the user's gain upon a swap.}

\newpage


\section{General properties of the swap function}

In the previous section, we defined properties regarding the gain of a user from a swap transaction. In this section we define some key properties regarding swap functions and what they imply on the AMMs. 

Output boundedness guarantees that an AMM has always enough output tokens 
$\tokT[1]$ to send to the user who performs a  
$\actAmmSwapExact{}{}{x}{\tokT[0]}{\tokT[1]}$.

\citeLean{AMMLib/Transaction/Swap/Rate.html\#SX.outputbound} % cite-lean(SX.outputbound)
\begin{defi}{Output-boundedness}{sr-output-bound}
  A swap rate function $\SX{}$ is \emph{output-bounded} when, for all $x,\ammR[0],\ammR[1]$ such that $x \geq 0$ and $\ammR[0], \ammR[1] > 0$:
  \[
  x \cdot \SX{x,\ammR[0],\ammR[1]} < \ammR[1]
  \]
\end{defi}

In other terms, this ensures that for any $\SX{}$ that satisfies this definition, the reserves of an already existing AMM will never be drained. 

Monotonicity relates the parameters of the swap rate function with its output. In particular, the output decreases if we decrease the amount of swapped tokens $x$, if we decrease the reserves of $\tokT[0]$ or if we increase the reserves of $\tokT[1]$. 

\citeLean{AMMLib/Transaction/Swap/Rate.html\#SX.strictmono} % cite-lean(SX.strictmono)
\begin{defi}{Strict monotonicity}{sr-strict-mono}
  A swap rate function $\SX{}$ is \emph{strictly monotonic} when,
  for $i \in \setenum{0,1,2}$ and $\lhd_i \in \setenum{<,\leq}$:
  \[
    x' \lhd_0 x,\; \ammRi[0] \lhd_1 \ammR[0],\; \ammR[1] \lhd_2 \ammRi[1]
    \implies
    \SX{x,\ammR[0],\ammR[1]}
    \lhd_3
    \SX{x',\ammRi[0],\ammRi[1]}
  \]
  where:
  \[
    \lhd_3 = \begin{cases}
      \leq & \text{if $\lhd_i = \,\leq$ for $i \in \setenum{0,1,2}$} \\
      < & \text{otherwise}
    \end{cases}
  \]
\end{defi}

This means that, for example, given a fixed input amount $x$ and a fixed reserve of $\tokT[0]$ $r_0$, a lower amount of reserves of $\tokT[1]$ $r_1$ will result in a higher swap rate output. This will be crucial on proving some of the lemmas presented further in the paper. 

The next crucial definition we want to introduce is the one of additivity, i.e. what is the output of two combined swaps. However, before doing so, we have to introduce a few auxiliary terms, since our definition of additivity slightly differs from the one introduced in \cite{BCL22lmcs}. The definition below for example, denotes the factor from which our definition of additivity differs from the original one.  

\marconote{aggiungere proof su questo e sui lemmi di zeta sotto}

\citeLean{AMMLib/FeeVersion/Additivity.html\#SX.fee.z_extended} % cite-lean(SX.fee.z_extended)
\begin{defi}{$\Zfee$}{fee:additive:zeta}
We define the function 

    \begin{align*}
        \Z x y {r_0} {r_1} = 
            \frac
            {\Big ( (\fee \ammR[1] x)(\ammR[0] + \fee x + \fee y) + (\fee \ammR[1] \ammR[0] y)\Big ) \cdot (\ammR[0] + x + \fee y)}
            {(\ammR[0] + \fee x + \fee y) \cdot \Big ( (\fee \ammR[1] x)(\ammR[0] + x + \fee y) + (\fee \ammR[1] \ammR[0] y)\Big )}
    \end{align*}

where $x, y, r_0, r_1 \in \mathbb{R}>0$
\end{defi}

Since this factor can be a bit too complex to work with, especially in some proofs, we introduce an equivalent factor in the lemma below, which is slightly easier to work with in some cases. 

\citeLean{AMMLib/FeeVersion/Additivity.html\#SX.fee.z_eq_z_extended} % cite-lean(SX.fee.z_eq_z_extended)
\begin{lemma}{$\Zfee$ reduced form}{fee:additive:zeta-reduced}
    $\forall \; x, y, r_0, r_1 \in \mathbb{R}>0$,

    \begin{align*}
        \Z x y {r_0} {r_1} =
            \frac
            {(x + y)(\ammR[0] + \fee x)(\ammR[0] + x + \fee y)}
            {(\ammR[0] + \fee x + \fee y)(x^2 + \ammR[0] x +\fee x y + \ammR[0] y)} 
    \end{align*}
\end{lemma}

It is also worth noting how this form is evidently less complex, as it does not depend at all on $r_1$. Next, it can be interesting to understand which values this term can assume, if it is always positive, negative or so on. The lemma below shows that this term not only always positive, but also always greater than one.

\marconote{magari conviene spostare questo più sotto dove è più facile spiegare perché serve? Così però non si vede che l'additività è "più grande"}

\citeLean{AMMLib/FeeVersion/Additivity.html\#SX.fee.z_factor_gt_1} % cite-lean(SX.fee.z_factor_gt_1)
\begin{lemma}{$\Zfee$ > 1}{fee:additive:zeta-gt-one}
     $\forall \; x, y, r_0, r_1 \in \mathbb{R}>0, \; \fee < 1 \implies \Z x y {r_0} {r_1} > 1$
\end{lemma}

Now we have everything to give our definition of additivity. 

\citeLean{AMMLib/FeeVersion/Additivity.html\#SX.fee.extended_additivity} % cite-lean(SX.fee.extended_additivity)
\begin{defi}{Additivity}{fee:sr-additivity}
  If a swap rate function $\SX{}$ is \emph{output-bounded} then it is also \emph{additive} when:  

 \[
    \SX{x+y,\ammR[0],\ammR[1]} = 
    \frac{\valSXa x + \valSXb y}{x+y} \cdot \Z x y {r_0} {r_1}
 \]
  where 
  \[
    \valSXa = \SX{x,\ammR[0],\ammR[1]},\;
    \valSXb = \SX{y,\ammR[0]+x,\ammR[1]-\valSXa x}
  \]
  % \bartnote{this does not seem part of a definition. Either change the definition into a theorem, or keep the definition and add a theorem. The first option is preferable if you do not use the term ``additive'' many times} 
  % \marconote{Fatto nel teorema ~\ref{lem:fee:additive:zeta-gt-one}, più simile a come ho formalizzato il tutto in Lean}
\end{defi}

Note how the term $\Z x y {r_0} {r_1}$ multiplies the fraction, which is exactly the definition of additivity provided in \cite{BCL22lmcs}. Since we know from Lemma~\ref{lem:fee:additive:zeta-gt-one} that this factor is always greater than one, this means that in a model that uses fees and with a swap rate function that satisfies this definition, a user is more likely to  swap larger sums of tokens rather than splitting this amount into two. 

The following lemma defines how additivity and gain are related to each other. In particular, the gain of two consecutive swap actions is given by the sum of the single gains plus a term $\epsilon_{\fee}$.
\newpage

\citeLean{AMMLib/FeeVersion/Additivity.html\#Swap.fee.additive_gain} % cite-lean(Swap.fee.additive_gain)
\begin{lemma}{Additivity of swap gain}{fee:swap-gain:additive}
  Let $\confG = \walA{\tokBal[\pmvA]} \mid \amm{\ammR[0]:\tokT[0]}{\ammR[1]:\tokT[1]} \mid \confD$. Let \mbox{$\txT(x) = \actAmmSwapExact{\pmvA}{}{x}{\tokT[0]}{\tokT[1]}$} denote the enabled transactions in $\confG$ for $x \in \mathbb{R}>0$
  and let $\confGi$ be such that, $\confG \xrightarrow{\txT(x_0)} \confGi$.
  
  If $\SX{}$ is \emph{output-bounded} and \emph{additive}, $\tokBal[\pmvA]\tokM{\tokT[0]}{\tokT[1]} = 0$ and $\fee < 1$ then:
  \[
    \gain[\confG]{\pmvA}{\txT(x_0+x_1)} 
    \; = \;
    \gain[\confG]{\pmvA}{\txT(x_0)} + \gain[\confGi]{\pmvA}{\txT(x_1)} + \epsilon_{\fee}
  \]
  where:
    \begin{align*}
        \epsilon_{\fee} & = \exchO{\tokT[1]}(\Z{x_0}{x_1}{r_0}{r_1}-1)(\valSXa x_0 + \valSXb x_1)
        \\
        \valSXa & = \SX{x_0, \ammR[0], \ammR[1]}
        \\
        \valSXb & = \SX{x_1, \ammR[0] + x_0, \ammR[1] - \valSXa x_0}
    \end{align*}
\end{lemma}

From the definition of $\epsilon_{\fee}$ it is also worth noting that this term is always positive under the lemma's assumptions, since all the terms are strictly positive and $\Z{x_0}{x_1}{r_0}{r_1} > 1$ from Lemma~\ref{lem:fee:additive:zeta-gt-one}. In the next section we will give a concrete example to show the reader this exact result using the constant product swap rate function.


\section{Properties of the costant-product swap function}
In this section we study in depth the constant product swap rate (Def~\ref{defi:const-prod}). As previously stated, this is the swap rate function that is mostly used in real-world AMMs implementations (such as Uniswap v2 \cite{uniswapimpl}), hence it is of particular interest analyzing the properties satisfied by this function. 

First of all, it is output-bounded, so AMMs that implement this function are guaranteed to have non-negative reserves upon any valid swap transaction. 

\citeLean{AMMLib/FeeVersion/Constprod.html\#SX.fee.constprod.outputbound} % cite-lean(SX.fee.constprod.outputbound)
\begin{lemma}{Constant product \emph{output-boundedness}}{fee:constprod:output-bound}
    The constant product swap rate function (Def.~\ref{defi:const-prod}) is output bounded (Def.~\ref{defi:sr-output-bound})
\end{lemma}
Then, the constant product swap rate function is also strictly monotonic. This also ensures that the internal rate is strictly monotonic, which is another key result later needed for the arbitrage proofs.

\citeLean{AMMLib/FeeVersion/Constprod.html\#SX.fee.constprod.strictmono} % cite-lean(SX.fee.constprod.strictmono)
\begin{lemma}{Constant product \emph{strict-monotonicity}}{fee:constprod:strict-mono}
    The constant product swap rate function (Def.~\ref{defi:const-prod}) is strictly monotonic (Def.~\ref{defi:sr-strict-mono})
\end{lemma}

Finally, we can also prove that the function is additive, based on the definition given above. 

\citeLean{AMMLib/FeeVersion/Constprod.html\#SX.fee.constprod.extended_additivity} % cite-lean(SX.fee.constprod.extended_additivity)
\begin{lemma}{Constant product \emph{additivity}}{fee:constprod:additivity}
The constant product swap rate function (Def.~\ref{defi:const-prod}) is additive (Def.~\ref{defi:fee:sr-additivity})
\end{lemma}

At first sight, the reader might think that the gain given by $x$ tokens on an AMM that uses the constant product swap rate function is the same as the sum of the gains of two consecutive swaps, where the sum of the swapped tokens is equal to $x$. However, we just proved that the constant product satisfies the definition of additivity given in Def.~\ref{defi:fee:sr-additivity}. So, if the trading fee $\fee$ is less than one, and the user performing the swap has no minted token, then it also means that the function satisfies all the assumption of Lemma~\ref{lem:fee:swap-gain:additive}. 

\marconote{aggiungere esempio per mostrare che la constprod fee non è additiva rispetto alla definizione originale}

Next, we present some lemmas that put in relation the swap rate, the internal and the external rate all together. This is of particular interest since it allows us to reason over the gain of the user, combining these results with Lemma~\ref{lem:fee:swap:gain-SX-X-fee}.

First, we analyze the relationship between the swap rate and the internal rate. 

\citeLean{AMMLib/FeeVersion/Constprod.html\#SX.fee.constprod.sx_rate_vs_int_rate} % cite-lean(SX.fee.constprod.sx_rate_vs_int_rate)
\begin{lemma}{Swap rate vs Internal rate}{fee:sx-rate-vs-int-rate}  
  Let $\confG = \walA{\tokBal[\pmvA]} \mid \amm{\ammR[0]:\tokT[0]}{\ammR[1]:\tokT[1]} \mid \confD$, $x \in \mathbb{R}>0$, 
  and $\confG \xrightarrow{\actAmmSwapExact{\pmvA}{}{x}{\tokT[0]}{\tokT[1]}} \confGi$. 
  %
  Then, if $\SX{}$ is the constant product swap rate (Def.~\ref{defi:const-prod}) and $\fee \leq 1$: 
  \begin{equation}
    \X[\confGi]{\tokT[0],\tokT[1]} < \SX{x, \ammR[0], \ammR[1]} < \X[\Gamma]{\tokT[0],\tokT[1]}
    \qquad
  \end{equation}
\end{lemma}
So we can see how the swap rate sits right between the internal rates computed in the AMM state before and after performing the swap transaction. It is also worth noting how this lemma is valid also when $\fee = 1$, which means that it would be valid also in the model presented in \cite{BCL22lmcs}.

Finally, we also want to analyze how the internal and external swap rate relate to each others. In particular, we consider a valid swap transaction and suppose that after this transaction, the external swap rate and the internal one are equal. This is a key assumption that we will use thoroughly when reasoning about the arbitrage later on.  
\albnote{Dato che $\txT(x) = \pmvA : \textit{swap}(x, \tokT[0], \tokT[1])$ viene usato in diversi posti, forse conviene introdurlo come notazione generale, invece che re-definirlo in ogni teorema che lo usa. Potrebbe aiutare a snellire un po' gli enunciati dei teoremi come nel'esempio di sopra.}

\marconote{Anche secondo me una definizione generale andrebbe meglio. Inoltre questa T(x) rappresenta solo le swap valide che può fare A}

\citeLean{AMMLib/FeeVersion/Constprod.html\#SX.fee.constprod.additive_int_rate_vs_ext_rate} % cite-lean(SX.fee.constprod.additive_int_rate_vs_ext_rate)
\begin{lemma}{Split Internal rate vs External rate}{fee:split-int-vs-ext-rate}
    Let $\confG = \walA{\tokBal[\pmvA]} \mid \amm{\ammR[0]:\tokT[0]}{\ammR[1]:\tokT[1]} \mid \confD$.
  For all $x \in \mathbb{R}>0$, 
  let $\txT(x) = \actAmmSwapExact{\pmvA}{}{x}{\tokT[0]}{\tokT[1]}$.
  Let $x_0$ be such that:
  \begin{equation}
    \label{eq:arbitrage:max:x0}
    % \SX{0,\ammR[0]+x_0,\ammR[1]-x_0 \cdot \SX{x_0,\ammR[0],\ammR[1]}}
    \X[\confGi]{\tokT[0],\tokT[1]} = \X{\tokT[0],\tokT[1]}
    \qquad
    \text{ where }
    \confG \xrightarrow{\txT(x_0)} \confGi
  \end{equation}
  Let $x_1, \; x_2 \in \mathbb{R}>0$ be such that $x_0 = x_1 + x_2$. Then, if $\SX{}$ is the constant product swap rate (~\ref{defi:const-prod}) and $\fee < 1$: 
  \begin{equation}
    \X{\tokT[0],\tokT[1]} < \X[\Gamma_2]{\tokT[0],\tokT[1]}
    \qquad
    \text{ where }
    \confG \xrightarrow{\txT(x_1)} \Gamma_1 \xrightarrow{\txT(x_2)} \Gamma_2
  \end{equation}
\end{lemma}

So in other terms, if we split the $x_0$ that brings the internal swap rate equal to the external one, and perform the two swaps, then the internal rate will be strictly greater than the external one. This is a result that we expected, since we already proved in Lemma~\ref{lem:fee:constprod:additivity} that splitting a swap transaction in two different swaps brings the AMM into a different state as well. 

\newpage

\section{Arbitrage}
In this section we want to present the final results of this paper. In particular, our main goal was to find the optimal amount of tokens that a user can swap given a certain state of the AMM that maximizes his gain. A first attempt at doing this could be mimicking the result found in \cite{BCL22lmcs}
, i.e. the optimal value is the one that brings the internal rate equal to the external one. We will ofter refer to this as the AMM reaching the "Equilibrium". 

The first challenge that arises is finding this particular value, which we specify in the Lemma below. 

\albnote{Se ci focalizziamo soltanto nella constant product, potremo considerare introdurre il lemma di sotto prima dei lemmi/teoremi che ipotizzano (senza svelare) l'esistenza e unicità della $x0$ che porta l'amm in equilibrio. Ad esempio, Th 2.10. Ma anche 2.9.}

\marconote{Sono d'accordo, secondo me dato che considero solo la constant product posso introdurre prima questo lemma, poi un lemma per dimostrare l'unicità di questo valore, e poi il lemma per l'equilibrio. Per ultimo invece il teorema del valore massimo}

\citeLean{AMMLib/FeeVersion/Arbitrage.html\#SX.fee.arbitrage.constprod.equil_value} % cite-lean(SX.fee.arbitrage.constprod.equil_value)
\begin{lemma}{Balance swap value}{arbitrage:balance}
  Let $\confG = \walA{\tokBal[\pmvA]} \mid \amm{\ammR[0]:\tokT[0]}{\ammR[1]:\tokT[1]}$,
  and let:
  \begin{equation}
    \label{eq:arbitrage:balance}
    x_0 
    \; = \;
    \frac
        {-\sqrt{\exchO{\tokT[0]}} \ammR[0] (1 + \fee) + \sqrt{\ammR[0]} \sqrt{\exchO{\tokT[0]} \ammR[0] (-1 + \fee)^2 + 4 \exchO{\tokT[1]} \ammR[1] \fee^2}}
        {2 \sqrt{\exchO{\tokT[0]}} \fee}
  \end{equation}
  If $\SX{}$ is the constant product swap rate, $x_0 \in \mathbb{R}>0$,   $\txT(x_0) = \actAmmSwapExact{\pmvA}{}{x_0}{\tokT[0]}{\tokT[1]}$ is a valid transaction and $\fee < 1$ then $\txT(x_0)$ bring the AMM to the balanced state, i.e. 
  \begin{align*}
      \X{\tokT[0], \tokT[1]} = \X[\confGi]{\tokT[0], \tokT[1]} 
      & \qquad \text{where} \; \confG \xrightarrow{\actAmmSwapExact{\pmvA}{}{x_0}{\tokT[0]}{\tokT[1]}} \confGi
  \end{align*}
\end{lemma}

Hence we know that this value, given the appropriate conditions, brings the AMM to the equilibrium. However, we are not sure, or better, we have not proved yet, that this is the only value that satisfies this condition. The following lemma states exactly this. 

\citeLean{AMMLib/FeeVersion/Arbitrage.html\#SX.fee.arbitrage.constprod.solution_equil_unique} % cite-lean(SX.fee.arbitrage.constprod.solution_equil_unique)
\begin{lemma}{Balance swap value uniqueness}{arbitrage:balance-unique}
    Let $\confG = \walA{\tokBal[\pmvA]} \mid \amm{\ammR[0]:\tokT[0]}{\ammR[1]:\tokT[1]}$,
  and let $x_0 \in \mathbb{R} > 0$ be s.t. $\txT(x_0) = \actAmmSwapExact{\pmvA}{}{x_0}{\tokT[0]}{\tokT[1]}$ is a valid transaction.
  If $\SX{}$ is the constant product swap rate and 
  \begin{align*}
      \X{\tokT[0], \tokT[1]} = \X[\confGi]{\tokT[0], \tokT[1]}
      & \qquad \text{where} \; \confG \xrightarrow{\actAmmSwapExact{\pmvA}{}{x_0}{\tokT[0]}{\tokT[1]}} \confGi
  \end{align*}
  then 
  \begin{align*}
      \neg \exists \; x_1 \in \mathbb{R}>0, x_1 \neq x_0 \; \land \;
      \X{\tokT[0], \tokT[1]} = \X[\confGii]{\tokT[0], \tokT[1]}
      & \qquad \text{where} \; \confG \xrightarrow{\actAmmSwapExact{\pmvA}{}{x_1}{\tokT[0]}{\tokT[1]}} \confGii
  \end{align*}
  i.e. $x_0$ is unique. 
\end{lemma}

From this lemma, we know that the value that brings the AMM to the equilibrium is unique. It is now natural to ask if such value is also the optimal value that maximizes the user's gain like it was for the previous model. However, this is not the case. 

\marconote{add an example here that computes the gain for the equilibrium x and for one slightly bigger, and show that is is better}. 

So, we know that this value is not the optimal one. However, it is very close to it, and it is still worth comparing the gain of this value to the one of every other value. The lemma below formalizes this comparison. 

\citeLean{AMMLib/FeeVersion/Arbitrage.html\#SX.fee.arbitrage.constprod.equil_value_solution_arbitrage} % cite-lean(SX.fee.arbitrage.constprod.equil_value_solution_arbitrage)
\begin{lemma}{Equilibrium value vs gain}{fee:equil-vs-gain}  
  Let $\confG = \walA{\tokBal[\pmvA]} \mid \amm{\ammR[0]:\tokT[0]}{\ammR[1]:\tokT[1]} \mid \confD$ 
  be such that $\tokBal[\pmvA]{\tokM{\tokT[0]}{\tokT[1]}} = 0$.
  For all $x > 0$, 
  let $\txT(x) = \actAmmSwapExact{\pmvA}{}{x}{\tokT[0]}{\tokT[1]}$.
  Let $x_0$ be the one defined in ~\ref{eq:arbitrage:balance} and assume that $x_0 \in \mathbb{R}>0$
  and  $\txT(x_0) = \actAmmSwapExact{\pmvA}{}{x_0}{\tokT[0]}{\tokT[1]}$ is a valid transaction.
  
  If $\SX{}$ is the constant product swap rate (~\ref{defi:const-prod}) and $\fee < 1$, then:
  \begin{align*}
    & \forall x < x_0 
    \; : \;
    \gain[{\confG}]{\pmvA}{\txT(x_0)}
    \; > \;
    \gain[{\confG}]{\pmvA}{\txT(x)}
    \\
    & \text{and}
    \\
    & \forall x > x_0 
    \; : \;
    (\gain[{\confG}]{\pmvA}{\txT(x_0)}
    \; > \;
    \gain[{\confG}]{\pmvA}{\txT(x)} \iff x > \frac{x_0}{\fee})
   \end{align*}
\end{lemma}

So we know that the optimal value is between $x_0$ and $\frac{x_0}{\fee}$. The only thing left to do is finding such value that maximizes the gain

\albnote{collegato al mio commento di sopra riguardo la notazione T(x), notare che sotto non viene usata. Meglio uniformare. }


\albnote{Sempre assumendo che ci concentriamo soltnato sul constant product, sotto farei riferimento allo x0 identificato da Lemma 2.11.}

\citeLean{AMMLib/FeeVersion/Maximal.html\#SX.fee.constprod.x_max_gain} % cite-lean(SX.fee.constprod.x_max_gain)

\begin{lemma}{Max Gain Value}{fee:max-gain}
  Let $\confG = \walA{\tokBal[\pmvA]} \mid \amm{\ammR[0]:\tokT[0]}{\ammR[1]:\tokT[1]}$ 
  be such that $\tokBal[\pmvA]{\tokM{\tokT[0]}{\tokT[1]}} = 0$.
  Let $x_0$ be such that:
  \begin{equation}
    \X[\confGi]{\tokT[0],\tokT[1]} = \X{\tokT[0],\tokT[1]}
    \qquad
    \text{ where }
    \confG \xrightarrow{\txT(x_0)} \confGi
  \end{equation}
  Let $x_{max}$ be
  \begin{equation}
    \label{eq:fee:arbitrage:max-value}
    x_{max} = x_0 + \frac
        {-\sqrt{\exchO{\tokT[0]}} \ammR[0] - \sqrt{\exchO{\tokT[0]}} \fee x_0 + \sqrt{\exchO{\tokT[1]} \fee \ammR[0] \ammR[1]}}
        {\sqrt{\exchO{\tokT[0]}} \fee}
  \end{equation}
  If $\SX{}$ is the constant product swap rate function and $\fee < 1$, then:
  \[
    \forall x \neq x_{max}
    \; : \;
    \gain[{\confG}]{\pmvA}{\txT(x_{max})}
    \; > \;
    \gain[{\confG}]{\pmvA}{\txT(x)}
  \]
\end{lemma}

Finally, it is also worth exploring if this is the only value that maximizes the gain, and this can be easily proved by contradiction, and stated in the lemma below. 

\newpage 

\citeLean{AMMLib/FeeVersion/Maximal.html\#SX.fee.constprod.x_max_unique} % cite-lean(SX.fee.constprod.x_max_unique)
\begin{lemma}{Optimal swap value uniqueness}{arbitrage:optimal-unique}
    Let $\confG = \walA{\tokBal[\pmvA]} \mid \amm{\ammR[0]:\tokT[0]}{\ammR[1]:\tokT[1]}$ be such that $\tokBal[\pmvA]{\tokM{\tokT[0]}{\tokT[1]}} = 0$.
   Let $x_0 \in \mathbb{R} > 0$ be s.t. $\txT(x_0) = \actAmmSwapExact{\pmvA}{}{x_0}{\tokT[0]}{\tokT[1]}$ is a valid transaction.
  If $\SX{}$ is the constant product swap rate, $\fee < 1$ and 
  \begin{align*}
      \X{\tokT[0], \tokT[1]} = \X[\confGi]{\tokT[0], \tokT[1]}
      & \qquad \text{where} \; \confG \xrightarrow{\actAmmSwapExact{\pmvA}{}{x_0}{\tokT[0]}{\tokT[1]}} \confGi
  \end{align*}
  then 
  \begin{align*}
      \neg \exists \; x_1 \in \mathbb{R}>0 \; : \;  x_1 \neq x_{max} \; \land \;
      \forall x \neq x_1
    \; : \;
    \gain[{\confG}]{\pmvA}{\txT(x_1)}
    \; > \;
    \gain[{\confG}]{\pmvA}{\txT(x)}
  \end{align*}
  i.e. $x_0$ is unique. 
\end{lemma}
