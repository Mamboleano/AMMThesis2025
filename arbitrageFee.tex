\section{Properties of swap with trading fees}

\marconote{test fixme}

\albnote{Nel paper LMCS/Lemma 3.2 c'è un caso per il gain di B diverso di A. Perché qui no? Non serve, assumo.}

\begin{lem}[Swap gain]
  \label{lem:fee:swap:gain}
  Let $\confG = \amm{\ammR[0]:\tokT[0]}{\ammR[1]:\tokT[1]} \mid \confD$,
  \bartnote{meglio specificare qui $\mid \walA{\tokBal[\pmvA]}$?}
  and let $\txT = \actAmmSwapExact{\pmvA}{}{x}{\tokT[0]}{\tokT[1]}$
  be enabled in $\confG$.
  Then:
  \begin{align*}
    \gain[{\confG}]{\pmvA}{\txT}
    & = \phantom{-}
      x \cdot \big(
      \SX{x,\ammR[0],\ammR[1]} \, \exchO{\tokT[1]}
      -
      \exchO{\tokT[0]}
      \big)
      \cdot
      \Big(
      1 - \frac{\tokBal[\pmvA]\tokM{\tokT[0]}{\tokT[1]}}{\supply[\confG]{\tokM{\tokT[0]}{\tokT[1]}}}
      \Big)
    && \text{if $\walA{\tokBal[\pmvA]} \in \confG$}
  \end{align*}


\bartnote{questo sembra un caso particolare, e lo metterei fuori dal lemma}
  Or, if $\tokBal[\pmvA]\tokM{\tokT[0]}{\tokT[1]} = 0$, 

  \begin{align*}
    \gain[{\confG}]{\pmvA}{\txT}
    & = \phantom{-}
      x \cdot \big(
      \SX{x,\ammR[0],\ammR[1]} \, \exchO{\tokT[1]}
      -
      \exchO{\tokT[0]}
      \big)
  \end{align*}
\end{lem}

\bartnote{l'uso del proofof è propedeutico a spostare tutte le proof in appendice?}

\begin{proofof}{Lemma}{lem:fee:swap:gain}
    Proof is the same as in~\cite{BCL22lmcs}.
\end{proofof}

\albnote{Quando aggiungerai il testo per guidare il lettore, conviene spiegare perché alcuni risultati escludono i  casi dove A non ha minted token.}

\begin{lem}[Swap rate \emph{vs.} exchange rate fee]
  \label{lem:fee:swap:gain-SX-X-fee}
  Let $\confG = \walA{\tokBal} \mid \amm{\ammR[0]:\tokT[0]}{\ammR[1]:\tokT[1]} \mid \confD$ 
  be such that $\tokBal{\tokM{\tokT[0]}{\tokT[1]}} = 0$, and
  let $\txT = \actAmmSwapExact{\pmvA}{}{x}{\tokT[0]}{\tokT[1]}$ 
  be enabled in $\confG$.
  Then: 
  \begin{align*}
      \gain[{\confG}]{\pmvA}{\txT} \circ 0
      \iff
      \SX{x,\ammR[0],\ammR[1]} \circ \X{\tokT[0],\tokT[1]}
       && \text{for $\circ \in \setenum{<, =, >}$}
  \end{align*}
\end{lem}

\begin{proofof}{Lemma}{lem:fee:swap:gain-SX-X-fee}
    Proof is the same as in~\cite{BCL22lmcs}.
\end{proofof}


\albnote{Sotto io avrei scritto l'equazione e poi ``where $\alpha$=...''}


\begin{defi}[Additivity]
  \label{def:fee:sr-additivity}
  A swap rate function $\SX{}$ is \emph{additive} when:
  \[
    \valSXa = \SX{x,\ammR[0],\ammR[1]},\;
    \valSXb = \SX{y,\ammR[0]+x,\ammR[1]-\valSXa x}
    \implies
    \SX{x+y,\ammR[0],\ammR[1]} = 
    \frac{\valSXa x + \valSXb y}{x+y} \cdot z
  \]

  where
  \bartnote{choose one of the two options}
  \bartnote{use $\zeta$ rather than $z$?}
  
  \begin{align*}
      z & = \frac
            {\Big ( (\fee \ammR[1] x)(\ammR[0] + \fee x + \fee y) + (\fee \ammR[1] \ammR[0] y)\Big ) \cdot (\ammR[0] + x + \fee y)}
            {(\ammR[0] + \fee x + \fee y) \cdot \Big ( (\fee \ammR[1] x)(\ammR[0] + x + \fee y) + (\fee \ammR[1] \ammR[0] y)\Big )}
    \\
    & \text{or}
    \\
    z & = \frac
            {(x + y)(\ammR[0] + \fee x)(\ammR[0] + x + \fee y)}
            {(\ammR[0] + \fee x + \fee y)(x^2 + \ammR[0] x +\fee x y + \ammR[0] y)} 
  \end{align*}

  \bartnote{this does not seem part of a definition. Either change the definition into a theorem, or keep the definition and add a theorem. The first option is preferable if you do not use the term ``additive'' many times}
  Also, if $\phi < 1$ then $z > 1$
\end{defi}

\albnote{Sotto parli di ogni $\SX{}$ o di una $\SX{}$ particolare (constant product)?}

\begin{proofof}{Theorem}{def:fee:sr-additivity}
    Here we prove that $\SX{}$ satisfies this definition.

    Let 
    \begin{align*}
        \valSXa & = \SX{x, \ammR[0], \ammR[1]} = \frac{\fee \ammR[1]}
                                                    {\ammR[0] + \fee x}
        \\
        \valSXb & = \SX{y, \ammR[0] + x, \ammR[1] - \valSXa x} = 
        \frac
            {\fee (\ammR[1] - \valSXa x)}
            {\ammR[0] + x + \fee y}
        = \frac 
            {\fee \cdot \Big ( \ammR[1] - \frac{\fee \ammR[1] x}{\ammR[0] + \fee x}\Big )}
            {\ammR[0] + x + \fee y}    
         = \frac 
            {\fee \Big ( \frac{\ammR[1](\ammR[0] + \fee x) -   \fee \ammR[1] x}{\ammR[0] + \fee x}\Big )}
            {\ammR[0] + x + \fee y}
        \\
        & = \frac{\fee (\ammR[1] \ammR[0] + \fee \ammR[1] x - \fee \ammR[1] x)}
                 {(\ammR[0] + \fee x)(\ammR[0] + x + \fee y)}
          = \frac{\fee \ammR[1] \ammR[0]}
                 {(\ammR[0] + \fee x)(\ammR[0] + x + \fee y)} 
    \end{align*}

    Then, 
    \begin{align*}
        \frac{\valSXa x + \valSXb y}{x+y} \cdot z 
        & = 
        \frac{1}{x + y} \Big ( 
            \frac{\fee \ammR[1] x}{\ammR[0] + \fee x} + 
            \frac{\fee \ammR[1] \ammR[0] y}{(\ammR[0] + \fee x)(\ammR[0] + x +\fee y)}
        \Big ) \cdot z
        \\
        & = 
        \frac{1}{x + y} \Big (
            \frac
                {\fee \ammR[0] \ammR[1] x + \fee \ammR[1] x^2 + \fee^2 \ammR[1] x y + \fee \ammR[0] \ammR[1] y}
                {(\ammR[0] + \fee x)(\ammR[0] + x +\fee y)}
        \Big ) \cdot z
        \\
        & = 
        \frac{1}{x + y} \Big (
            \frac
                {(\fee \ammR[1] x)(\ammR[0] + x + \fee y) + (\fee \ammR[1] \ammR[0] y)}
                {(\ammR[0] + \fee x)(\ammR[0] + x +\fee y)}
        \Big ) \cdot 
        \\
        & \frac
            {\Big ( (\fee \ammR[1] x)(\ammR[0] + \fee x + \fee y) + (\fee \ammR[1] \ammR[0] y)\Big ) \cdot (\ammR[0] + x + \fee y)}
            {(\ammR[0] + \fee x + \fee y) \cdot \Big ( (\fee \ammR[1] x)(\ammR[0] + x + \fee y) + (\fee \ammR[1] \ammR[0] y)\Big )}
        \\
        & = 
        \frac{1}{x + y}
            \frac
                {(\fee \ammR[1] x)(\ammR[0] + \fee x + \fee y) + (\fee \ammR[1] \ammR[0] y)}
                {(\ammR[0] + \fee x)(\ammR[0] + \fee x +\fee y)}
        \\        
        & = 
        \frac{1}{x + y}
            \frac
                {\fee \ammR[0] \ammR[1] x + \fee^2 \ammR[1] x^2 + \fee^2 \ammR[1] x y + \fee \ammR[1] \ammR[0] y}
                {(\ammR[0] + \fee x)(\ammR[0] + \fee x +\fee y)}
        \\        
        & = 
        \frac{1}{x + y}
            \frac
                {\fee \ammR[1] (\ammR[0] x + \fee x^2 + \fee x y + \ammR[0] y)}
                {(\ammR[0] + \fee x)(\ammR[0] + \fee x +\fee y)}
        \\        
        & = 
        \frac{1}{x + y}
            \frac
                {\fee \ammR[1] (\ammR[0] + \fee x)(x + y)}
                {(\ammR[0] + \fee x)(\ammR[0] + \fee x +\fee y)}
        \\        
        & = 
        \frac{\fee \ammR[1]}{\ammR[0] + \fee (x + y)}
        \\
        & = 
        \SX{x+y, \ammR[0], \ammR[1]}
    \end{align*}

    Now we prove that $z > 1$
    \begin{align*}
        & \frac
        {(x + y)(\ammR[0] + \fee x)(\ammR[0] + x + \fee y)}
        {(\ammR[0] + \fee x + \fee y)(x^2 + \ammR[0] x +\fee x y + \ammR[0] y)} - 1 > 0
        \\
        \iff & (x + y)(\ammR[0] + \fee x)(\ammR[0] + x + \fee y) - (\ammR[0] + \fee x + \fee y)(x^2 + \ammR[0] x +\fee x y + \ammR[0] y) > 0
        \\
        \iff & \ammR[0] x y + x(\ammR[0] + \phi x)(\ammR[0] + x + \phi y) + \ammR[0] y (\ammR[0] + \phi y) + \phi x y (\ammR[0] + x + \phi y)
        \\
        & - \phi \ammR[0] x y - \phi x (x^2 + \ammR[0] x + \phi x y) - (\ammR[0] + \phi y)(x^2 + \ammR[0] x + \phi x y + \ammR[0] y) > 0
        \\
        \iff & \ammR[0] x y (1 - \phi) + (\ammR[0] + \phi x) (x^2 + \ammR[0] x + \phi x y) + \phi y (x^2 + \ammR[0] x + \phi x y) + \ammR[0] y (\ammR[0] + \phi y)
        \\
        & - \phi x (x^2 + \ammR[0] x + \phi x y) - \ammR[0] y (\ammR[0] + \phi y) - (\ammR[0] + \phi y)(x^2 + \ammR[0] x + \phi x y) > 0
        \\
        \iff & \ammR[0] x y (1 - \phi) + (x^2 + \ammR[0] x + \phi x y)(\ammR[0] + \phi x + \phi y - \phi x - \ammR[0] - \phi y) > 0
        \\
        \iff & \ammR[0] x y (1 - \phi) > 0
        \\
        \iff & 1 - \phi > 0
        \\
        \iff & \phi < 1
    \end{align*}
\end{proofof}

\begin{thm}[Additivity of swap]
  \label{thm:fee:sr-additivity}
  Let $\confG \xrightarrow{\txT[0]} \confG[0] \xrightarrow{\txT[1]} \confG[1]$,
  with
  \mbox{$\txT[i] = \actAmmSwapExact{\pmvA}{}{x_i}{\tokT[0]}{\tokT[1]}$} 
  for $i \in \setenum{0,1}$.
  If $\SX{}$ is additive, then:
  \[
    \confG \xrightarrow{\actAmmSwapExact{\pmvA}{}{x_0+x_1}{\tokT[0]}{\tokT[1]}} \confG[1]
  \]

  This theorem does not hold!
  \bartnote{then remove the theorem environment, and replace it with an example or a remark}
\end{thm}


\bartnote{added}
Output boundedness guarantees that an AMM has always enough output tokens 
$\tokT[1]$ to send to the user who performs a  
$\actAmmSwapExact{}{}{x}{\tokT[0]}{\tokT[1]}$.

\begin{defi}[Output-boundedness]
  \label{def:sr-output-bound}
  A swap rate function $\SX{}$ is \emph{output-bounded} when, for all $x,\ammR[0],\ammR[1]$ such that $x \geq 0$ and $\ammR[0], \ammR[1] > 0$:
  \[
  x \cdot \SX{x,\ammR[0],\ammR[1]} < \ammR[1]
  \]
\end{defi}


\begin{lem}[Additivity of swap gain fee]
  \label{lem:fee:swap-gain:additive}
  Let \mbox{$\txT(x) = \actAmmSwapExact{\pmvA}{}{x}{\tokT[0]}{\tokT[1]}$},
  and let $\confG \xrightarrow{\txT(x_0)} \confGi$.
  If $\SX{}$ is output-bounded and additive, and $\tokBal[\pmvA]\tokM{\tokT[0]}{\tokT[1]} = 0$ then:
  \[
    \gain[\confG]{\pmvA}{\txT(x_0+x_1)} 
    \; = \;
    \gain[\confG]{\pmvA}{\txT(x_0)} + \gain[\confGi]{\pmvA}{\txT(x_1)} + \epsilon_{\fee}
  \]
  where:
  \[
    \epsilon_{\fee} = \exchO{\tokT[1]}(z-1)(\valSXa x_0 + \valSXb x_1)
  \]
    \bartnote{here $z$ is not defined: it seems that $z$ is similar to that in the definition of additivity, not slightly different? The same for $\alpha$ and $\beta$. Define $\alpha$, $\beta$ and $\zeta$ as functions of $x, y, r_0, r_1$?}

    and:
    \begin{align*}
        \valSXa & = \SX{x_0, \ammR[0], \ammR[1]}
        \\
        \valSXb & = \SX{x_1, \ammR[0] + x_0, \ammR[1] - \valSXa x_0}
    \end{align*}
\end{lem}

\albnote{I assume that blank steps are just some arithmetic manipulations/simplifications.}

\begin{proofof}{Theorem}{lem:fee:swap-gain:additive}
By Definition~\ref{def:fee:sr-additivity} we have that
    \begin{align*}
        \valSXc = \SX{x_0+x_1, \ammR[0], \ammR[1]} = 
        \frac{\valSXa x_0 + \valSXa x_1}{x_0 + x_1} \cdot z
    \end{align*}
    So, 
    \begin{align*}
        & \gain[\confG]{\pmvA}{\txT(x_0 + x_1)} - \gain[\confG]{\pmvA}{\txT(x_0)}
        \\
        & = \valSXc (x_0 + x_1)\exchO{\tokT[1]} - (x_0 + x_1) \exchO{\tokT[0]} 
          - \valSXa x_0 \exchO{\tokT[1]} + x_0 \exchO{\tokT[0]}
        \\
        & = (\valSXc (x_0 + x_1) - \valSXa x_0) \exchO{\tokT[1]} - x_1 \exchO{\tokT[0]}
        \\
        & = (z \valSXa x_0 + z \valSXb x_1 - \valSXa x_0) \exchO{\tokT[1]} - x_1 \exchO{\tokT[0]}
        \\
        & = z \valSXa x_0 \exchO{\tokT[1]} + z \valSXb x_1 \exchO{\tokT[1]} - \valSXa x_0 \exchO{\tokT[1]} - x_1 \exchO{\tokT[0]}
        \\
        & = z \valSXa x_0 \exchO{\tokT[1]} + z \valSXb x_1 \exchO{\tokT[1]} - \valSXa x_0 \exchO{\tokT[1]} - x_1 \exchO{\tokT[0]} + \valSXb x_1 \exchO{\tokT[1]} - \valSXb x_1 \exchO{\tokT[1]}
        \\
        & = (\valSXb x_1 \exchO{\tokT[1]} - x_1 \exchO{\tokT[0]}) + (z \valSXa x_0 \exchO{\tokT[1]} + z \valSXb x_1 \exchO{\tokT[1]} - \valSXa x_0 \exchO{\tokT[1]} - \valSXb x_1 \exchO{\tokT[0]})
        \\
        & = \gain[\confGi]{\pmvA}{\txT(x_1)} + \exchO{\tokT[1]}(z-1)(\valSXa x_0 + \valSXb x_1) && (\text{by Def.~\ref{lem:fee:swap:gain}})
        \\
        & = \gain[\confGi]{\pmvA}{\txT(x_1)} + \epsilon_{\fee} && (\text{by Def. of $\epsilon_{\fee}$})
    \end{align*}
\end{proofof}

\begin{thm}[Helper 3]
  \label{thm:fee:Helper3}  
  Let $\confG = \walA{\tokBal} \mid \amm{\ammR[0]:\tokT[0]}{\ammR[1]:\tokT[1]} \mid \confD$ 
  be such that $\tokBal{\tokM{\tokT[0]}{\tokT[1]}} = 0$.
  For all $x > 0$, 
  let $\txT(x) = \actAmmSwapExact{\pmvA}{}{x}{\tokT[0]}{\tokT[1]}$.
  Take $x_0 > 0$, then:
  \begin{equation}
    \X[\confGi]{\tokT[0],\tokT[1]} < \SX{x_0, \ammR[0], \ammR[1]} < \X[\Gamma]{\tokT[0],\tokT[1]}
    \qquad
    \text{ where }
    \confG \xrightarrow{\txT(x_0)} \confGi
  \end{equation}
  
\end{thm}

\albnote{in qualche meeting precedente mi è sembrato di sentire che il vincolo $\tokBal{\tokM{\tokT[0]}{\tokT[1]}} = 0$ non serve (in uno dei teoremi, non ricordo quale). Nel caso, controllare.}


\albnote{La mia versione (direi più snella) del Helper 3 sotto. }

\begin{thm}[Helper 3]
  \label{thm:fee:Helper3}  
  Let $\confG = \walA{\tokBal} \mid \amm{\ammR[0]:\tokT[0]}{\ammR[1]:\tokT[1]} \mid \confD$ 
  be such that $\tokBal{\tokM{\tokT[0]}{\tokT[1]}} = 0$, $x > 0$, 
  and $\confG \xrightarrow{\actAmmSwapExact{\pmvA}{}{x}{\tokT[0]}{\tokT[1]}} \confGi$. 
  %
  Then: 
  \begin{equation}
    \X[\confGi]{\tokT[0],\tokT[1]} < \SX{x, \ammR[0], \ammR[1]} < \X[\Gamma]{\tokT[0],\tokT[1]}
    \qquad
  \end{equation}
\end{thm}


\begin{proofof}{Theorem}{thm:fee:Helper3} 
    \marconote{Proof wrong in the second part, fixed it in Lean}
    We split the proof in two parts: 
    \begin{itemize}
        \item $\SX{x_0, \ammR[0], \ammR[1]} < \X[\Gamma]{\tokT[0],\tokT[1]}$: 
        \begin{align*}
            \X[\confG]{\tokT[0],\tokT[1]} 
                & = \lim_{z \rightarrow 0} \SX{z,\ammR[0],\ammR[1]} && \text{(by Def. of X)}
                \\
                & > \SX{x_0, \ammR[0], \ammR[1]}                    && \text{Strict. Mono.}
        \end{align*}

        \item $\X[\confGi]{\tokT[0],\tokT[1]} < \SX{x_0, \ammR[0], \ammR[1]}$:

            Let $\valSXa = \SX{x_0, \ammR[0], \ammR[1]}$. 
            
            Before continuing the proof, lets first prove that
                \begin{equation}
                    \label{eq:SX-less-SX-fee}
                    \frac{\fee(\ammR[1] - \valSXa x_0)}{\ammR[0] + x_0} < \frac{\fee \ammR[1]}{\ammR[0] + x_0}
                \end{equation}
                \begin{proof}
                    \begin{align*}
                        & \frac{\fee(\ammR[1] - \valSXa x_0)}{\ammR[0] + x_0} < \frac{\fee 
                          \ammR[1]}{\ammR[0] + x_0}    && \text{(Den. is positive)}
                        \\
                        \iff & \fee \ammR[1] - \fee \valSXa x_0 - \fee \ammR[1] < 0
                        \\
                        \iff & - \fee \valSXa x_0 < 0
                    \end{align*}
                    Since all the factors are positive, then it is true.
                \end{proof}
                Now we can continue the proof:
                \begin{align*}
                    \X[\confGi]{\tokT[0],\tokT[1]} 
                    & = 
                    \lim_{z \rightarrow 0} \SX{z,\ammR[0] + x_0,\ammR[1] - \valSXa x_0} && \text{(by Def. of X)}
                    \\
                    & =
                    \frac{\fee(\ammR[1] - \alpha x_0)}{\ammR[0] + x_0}
                    \\
                    & < 
                    \frac{\fee \ammR[1]}{\ammR[0] + x_0} && \text{by Eq~\ref{eq:SX-less-SX-fee}}
                    \\
                    & = 
                    \SX{x_0, \ammR[0], \ammR[1]}
                \end{align*}
    \end{itemize}
\end{proofof}

\albnote{Dato che $\txT(x) = \actAmmSwapExact{\pmvA}{}{x}{\tokT[0]}{\tokT[1]}$ viene usato in diversi posti, forse conviene introdurlo come notazione generale, invece che re-definirlo in ogni teorema che lo usa. Potrebbe aiutare a snellire un po' gli enunciati dei teoremi come nel'esempio di sopra.}

\begin{thm}[Helper 4]
  \label{thm:fee:helper-4}
    Let $\confG = \walA{\tokBal} \mid \amm{\ammR[0]:\tokT[0]}{\ammR[1]:\tokT[1]} \mid \confD$ 
  be such that $\tokBal{\tokM{\tokT[0]}{\tokT[1]}} = 0$.
  For all $x > 0$, 
  let $\txT(x) = \actAmmSwapExact{\pmvA}{}{x}{\tokT[0]}{\tokT[1]}$.
  Let $x_0$ be such that:
  \begin{equation}
    \label{eq:arbitrage:max:x0}
    % \SX{0,\ammR[0]+x_0,\ammR[1]-x_0 \cdot \SX{x_0,\ammR[0],\ammR[1]}}
    \X[\confGi]{\tokT[0],\tokT[1]} = \X{\tokT[0],\tokT[1]}
    \qquad
    \text{ where }
    \confG \xrightarrow{\txT(x_0)} \confGi
  \end{equation}
  Let $x_1 > 0 $ and $x_2 > 0$ be such that $x_0 = x_1 + x_2$, then: 
  \begin{equation}
    \X{\tokT[0],\tokT[1]} < \X[\Gamma_2]{\tokT[0],\tokT[1]}
    \qquad
    \text{ where }
    \confG \xrightarrow{\txT(x_1)} \Gamma_1 \xrightarrow{\txT(x_2)} \Gamma_2
  \end{equation}
\end{thm}

\begin{proofof}{Theorem}{thm:fee:helper-4}
    Consider
    \begin{align*}
        y_1 & = \SX{x_1, \ammR[0], \ammR[1]} \cdot x_1 
        \\  & = \frac{\fee \ammR[1] x_1}{\ammR[0] + \fee x_1}
        \\
        y_2 & = \SX{x_2, \ammR[0] + x_1, \ammR[1] - y_1} \cdot x_2
        \\  & = \frac{\fee \ammR[0] \ammR[1] x_2}{(\ammR[0] + \fee x_1)(\ammR[0] + x_1 + \fee x_2)}
        \\
        y_3 & = \SX{x1 + x_2, \ammR[0] + x_1 + x_2, \ammR[1] - (y_1 + y_2)} \cdot (x_1 + x_2)
        \\  & = \SX{x_0, \ammR[0] + x_0, \ammR[1] - (y_1 + y_2)} \cdot x_0 
        \\  & = \frac{\fee \ammR[1] (x_1 + x_2)}{\ammR[0] + \fee x_1 + \fee x_2}
    \end{align*}

    Then, 
    \begin{align*}
        \X[{\confG[2]}]{\tokT[0],\tokT[1]} & = 
        \lim_{z \rightarrow 0} \SX{z,\ammR[0] + x_0,\ammR[1] - (y_1 + y_2)}
        && (\text{By Def.})
        \\
        & > \lim_{z \rightarrow 0} \SX{z,\ammR[0] + x_0,\ammR[1] - y_3}
        && (\text{By strict mono. proof below})
        \\
        & = \X[\confGi]{\tokT[0],\tokT[1]} = \X{\tokT[0],\tokT[1]}
        && (\text{By. Def. and Hyp.})
    \end{align*}

    We know that $z \leq z$ and $\ammR[0] + x_0 \leq \ammR[0] + x_0$, hence we only have to prove that $\ammR[1] - (y_1 + y_2) > \ammR[1] - y_3$:
    \begin{align*}
        & \ammR[1] - (y_1 + y_2) > \ammR[1] - y_3
        \\
        \iff & y_1 + y_2 - y_3 < 0
        \\
        \iff & \frac{\fee \ammR[1] x_1}{\ammR[0] + \fee x_1} + \frac{\fee \ammR[0] \ammR[1] x_2}{(\ammR[0] + \fee x_1)(\ammR[0] + x_1 + \fee x_2)}
        - \frac{\fee \ammR[1] (x_1 + x_2)}{\ammR[0] + \fee x_1 + \fee x_2} < 0
        \\
        \iff & \frac
            {\fee \ammR[1] x_1 (\ammR[0] + x_1 + \fee x_2)(\ammR[0] + \fee x_1 + \fee x_2) + \fee \ammR[0] \ammR[1] x_2 (\ammR[0] + \fee x_1 + \fee x_2)}
            {(\ammR[0] + \fee x_1)(\ammR[0] + x_1 + \fee x_2)(\ammR[0] + \fee x_1 + \fee x_2)}
            \\
            - & \frac{\fee \ammR[1] x_1 (\ammR[0] + \fee x_1)(\ammR[0] + x_1 + \fee x_2) + \fee \ammR[1] x_2 (\ammR[0] + \fee x_1)(\ammR[0] + x_1 + \fee x_2)}
            {(\ammR[0] + \fee x_1)(\ammR[0] + x_1 + \fee x_2)(\ammR[0] + \fee x_1 + \fee x_2)} < 0
        \\
        \iff & 
            \fee \ammR[1] x_1 (\ammR[0] + x_1 + \fee x_2)(\ammR[0] + \fee x_1 + \fee x_2) + \fee \ammR[0] \ammR[1] x_2 (\ammR[0] + \fee x_1 + \fee x_2) 
            \\
            - & \fee \ammR[1] x_1 (\ammR[0] + \fee x_1)(\ammR[0] + x_1 + \fee x_2) - \fee \ammR[1] x_2 (\ammR[0] + \fee x_1)(\ammR[0] + x_1 + \fee x_2) < 0
        \\
        \iff & 
            \fee \ammR[1] x_1 (\ammR[0] + x_1 + \fee x_2)((\ammR[0] + \fee x_1 + \fee x_2) - (\ammR[0] + \fee x_1)) + \fee \ammR[0] \ammR[1] x_2 (\ammR[0] + \fee x_2)
            \\
            + & \fee^2 \ammR[0] \ammR[1] x_1 x_2 - \fee \ammR[0] \ammR[1] x_1 x_2 - \fee \ammR[0] \ammR[1] x_2 (\ammR[0] + \fee x_2) - \fee^2 \ammR[1] x_1 x_2 (\ammR[0] + x_1 + \fee x_2) < 0
        \\
        \iff & 
            \fee^2 \ammR[1] x_1 x_2 (\ammR[0] + x_1 + \fee x_2) + \fee \ammR[0] \ammR[1] x_2 (\ammR[1] + \fee x_2) + \fee^2 \ammR[0] \ammR[1] x_1 x_2 - \fee \ammR[0] \ammR[1] x_1 x_2
            \\
            - & \fee \ammR[0] \ammR[1] x_2 (\ammR[0] + \fee x_2) - \fee^2 \ammR[1] x_1 x_2 (\ammR[0] + x_1 + \fee x_2) < 0
        \\
        \iff & 
            \fee^2 \ammR[0] \ammR[1] x_1 x_2 - \fee \ammR[0] \ammR[1] x_1 x_2 < 0
        \\
        \iff & 
            \fee \ammR[0] \ammR[1] x_1 x_2 (\fee - 1) < 0
        \\
        \iff &
            \fee - 1 < 0 
        \\
        \iff & \fee < 1
    \end{align*}
\end{proofof}

\begin{thm}[ArbitrageFee]
  \label{thm:fee:arbitrage}  
  Let $\confG = \walA{\tokBal} \mid \amm{\ammR[0]:\tokT[0]}{\ammR[1]:\tokT[1]} \mid \confD$ 
  be such that $\tokBal{\tokM{\tokT[0]}{\tokT[1]}} = 0$.
  For all $x > 0$, 
  let $\txT(x) = \actAmmSwapExact{\pmvA}{}{x}{\tokT[0]}{\tokT[1]}$.
  Let $x_0$ be such that:
  \begin{equation}
    \label{eq:fee:arbitrage:max:x0}
    % \SX{0,\ammR[0]+x_0,\ammR[1]-x_0 \cdot \SX{x_0,\ammR[0],\ammR[1]}}
    \X[\confGi]{\tokT[0],\tokT[1]} = \X{\tokT[0],\tokT[1]}
    \qquad
    \text{ where }
    \confG \xrightarrow{\txT(x_0)} \confGi
  \end{equation}
  If $\fee < 1$, $\SX{}$ is output-bounded, strictly monotonic, additive and satisfies Helpers 3 and 4, then:
  \[
    \forall x \neq x_0
    \; : \;
    \gain[{\confG}]{\pmvA}{\txT(x_0)}
    \; > \;
    \gain[{\confG}]{\pmvA}{\txT(x)}
  \]

  Furthermore, if such an $x_0$ exists, then it is unique. 
\end{thm}

\begin{proofof}{Theorem}{thm:fee:arbitrage}
    \begin{itemize}
        \item Case $x < x_0$: 

            Let $x_1 > 0$ be such that $x_0 = x + x_1$. Since $\SX{}$ is output bounded, then $\txT(x_0)$, $\txT(x)$ and $\txT(x_1)$ are all enabled, in particular : 

            \begin{align*}
                & \confG \xrightarrow{\quad\quad\txT(x_0)\quad\quad} \confGi
                \\
                & \confG \xrightarrow{\txT(x)} \confG[1]\xrightarrow{\txT(x_1)} \confG[2]
            \end{align*}

            By Lemma ~\ref{lem:fee:swap-gain:additive} we know that: 

            \[
            \gain[\confG]{\pmvA}{\txT(x_0)} 
            \; = \;
            \gain[\confG]{\pmvA}{\txT(x)} + \gain[{\confG[1]}]{\pmvA}{\txT(x_1)} + \epsilon_{\fee}
            \]

            Since we know that $\epsilon_{\fee} > 0$ we just have to prove that $\gain[{\confG[1]}]{\pmvA}{\txT(x_1)} > 0$. To do that, we first prove that $\SX{x_1, \ammR[0] + x, \ammR[1] - \valSXa x} > \X{\tokT[0], \tokT[1]}$ where $\valSXa = \SX{x, \ammR[0], \ammR[1]}$: 
            \begin{align*}
                    \X{\tokT[0], \tokT[1]}
                & < \X[{\confG[2]}]{\tokT[0], \tokT[1]} && \text{by ~\ref{thm:fee:helper-4}}
                \\
                & < \SX{x_1, \ammR[0] + x, \ammR[1] - \valSXa x}    && \text{by ~\ref{thm:fee:Helper3}}
            \end{align*}

            So, by applying Lemma ~\ref{lem:fee:swap:gain-SX-X-fee} we can finish the proof.

        \item Case $x > x_0$: 
             Let $x_1 > 0$ be such that $x = x_0 + x_1$. Since $\SX{}$ is output bounded, then $\txT(x_0)$, $\txT(x)$ and $\txT(x_1)$ are all enabled, in particular : 

            \begin{align*}
                & \confG \xrightarrow{\quad\quad\txT(x)\quad\quad} \confG[1]
                \\
                & \confG \xrightarrow{\txT(x_0)} \confGi\xrightarrow{\txT(x_1)} \confG[2]
            \end{align*}

            By Lemma ~\ref{lem:fee:swap-gain:additive} we know that: 

            \[
            \gain[\confG]{\pmvA}{\txT(x)} 
            \; = \;
            \gain[\confG]{\pmvA}{\txT(x_0)} + \gain[{\confGi}]{\pmvA}{\txT(x_1)} + \epsilon_{\fee}
            \]

            To finish our proof, we need to prove that $\gain[{\confGi}]{\pmvA}{\txT(x_1)} + \epsilon_{\fee} < 0$. Since we know that $\epsilon_{\fee} > 0$, we have first to see if $\gain[{\confGi}]{\pmvA}{\txT(x_1)} < 0$, otherwise it is always false. So, we want to show that $\SX{x_1, \ammR[0] + x, \ammR[1] - \valSXa x} < \X{\tokT[0], \tokT[1]}$ where $\valSXa = \SX{x_0, \ammR[0], \ammR[1]}$: 
                \begin{align*}
                    \X{\tokT[0], \tokT[1]}
                & = \X[{\confGi}]{\tokT[0], \tokT[1]}
                \\
                & < \SX{x_1, \ammR[0] + x_0, \ammR[1] - \valSXa x_0}    && \text{by ~\ref{thm:fee:Helper3}}
            \end{align*}

            So, by applying Lemma ~\ref{lem:fee:swap:gain-SX-X-fee} we know that $\gain[{\confGi}]{\pmvA}{\txT(x_1)} < 0$, hence we can continue the proof.

            \begin{proof}
            
                Before proceeding with the proof, we point out the values of some terms that we will be using during the proof: 

                \begin{align*}
                    x_1 \valSXb
                    & = x_1 \cdot \frac
                            {\fee \Big ( \ammR[1] - \frac
                                {\fee \ammR[1]}
                                {\ammR[0] + \fee x_0} \cdot x_0
                            \Big )}
                            {\ammR[0] + x_0 + \fee x_1 }
                      = x_1 \cdot \frac
                            {\fee \Big (\frac
                                {\ammR[1] (\ammR[0] + \fee x_0) - \fee \ammR[1] x_0}
                                {\ammR[0] + \fee x_0}
                            \Big )}
                            {\ammR[0] + x_0 + \fee x_1 }
                    \\
                    & = x_1 \cdot \frac
                            {\fee \Big (\frac
                                {\ammR[1] \ammR[0] + \fee \ammR[1] x_0 - \fee \ammR[1] x_0}
                                {\ammR[0] + \fee x_0}
                            \Big )}
                            {\ammR[0] + x_0 + \fee x_1 }
                     = \frac
                            {\fee \ammR[0] \ammR[1] x_1}
                            {(\ammR[0] + x_0 + \fee x_1) (\ammR[0] + \fee x_0)}
                \\
                \\
                    (\valSXa x_0 + \valSXb x_1)
                    & = 
                    \frac{\fee \ammR[1] x_0}{\ammR[0] + \fee x_0} 
                    + 
                    \frac
                        {\fee \ammR[0] \ammR[1] x_1}
                        {(\ammR[0] + x_0 + \fee x_1) (\ammR[0] + \fee x_0)}
                    = 
                    \frac
                        {\fee \ammR[1] x_0 (\ammR[0] + x_0 + \fee x_1) + \fee \ammR[0] \ammR[1] x_1}
                        {(\ammR[0] + x_0 + \fee x_1) (\ammR[0] + \fee x_0)}
                    \\
                    & = 
                    \frac
                        {\fee \ammR[0] \ammR[1] x_0 + \fee \ammR[1] x_0^2 + \fee^2 \ammR[1] x_0 x_1 + \fee \ammR[0] \ammR[1] x_1} 
                        {(\ammR[0] + x_0 + \fee x_1) (\ammR[0] + \fee x_0)}
                    = 
                    \frac
                        {\fee \ammR[1] (\ammR[0] x_0 + x_0^2 + \fee x_0 x_1 + \ammR[0] x_1)} 
                        {(\ammR[0] + x_0 + \fee x_1) (\ammR[0] + \fee x_0)}
                \\
                \\
                    (z-1)
                    & = \frac
                        {(x_0 + x_1)(\ammR[0] + \fee x_0)(\ammR[0] + x_0 + \fee x_1)}
                        {(\ammR[0] + \fee x_0 + \fee x_1)(x_0^2 + \ammR[0] x_0 +\fee x_0 x_1 + \ammR[0] x_1)} - 1
                    \\
                    & = \frac
                        {(\ammR[0] x_0 + \fee x_0 ^ 2 + \ammR[0] x_1 + \fee x_0 x_1)(\ammR[0] + x_0 + \fee x_1)}
                        {(\ammR[0] + \fee x_0 + \fee x_1)(x_0^2 + \ammR[0] x_0 +\fee x_0 x_1 + \ammR[0] x_1)}
                    \\
                    & -
                        \frac
                        {(\ammR[0] + \fee x_0 + \fee x_1 )(\ammR[0] x_0 + x_0 ^ 2 + \ammR[0] x_1 + \fee x_0 x_1)}
                        {(\ammR[0] + \fee x_0 + \fee x_1)(x_0^2 + \ammR[0] x_0 +\fee x_0 x_1 + \ammR[0] x_1)}
                    \\
                    & = - \frac
                        {\ammR[0] x_0 x_1 (\fee - 1)}
                        {(\ammR[0] + \fee x_0 + \fee x_1)(x_0^2 + \ammR[0] x_0 +\fee x_0 x_1 + \ammR[0] x_1)}
                \\
                \\
                    x_1 \valSXb & + (z - 1)(\valSXa x_0 + \valSXb x_1)
                    \\
                    & = \frac
                            {\fee \ammR[0] \ammR[1] x_1}
                            {(\ammR[0] + x_0 + \fee x_1) (\ammR[0] + \fee x_0)} 
                            - \frac
                            {\ammR[0] \ammR[1] x_0 x_1 (\fee - 1) \fee}
                            {(\ammR[0] + \fee x_0 + \fee x_1) (\ammR[0] + \fee x_0) ( \ammR[0] + x_0 + \fee x_1)}
                    \\
                    & = 
                        \frac
                            {\fee \ammR[0] \ammR[1] x_1 (\ammR[0] + \fee x_0 + \fee x_1) - \ammR[0] \ammR[1] x_0 x_1 (\fee - 1) \fee}
                            {(\ammR[0] + \fee x_0 + \fee x_1) (\ammR[0] + \fee x_0) ( \ammR[0] + x_0 + \fee x_1)}
                    \\
                    & = 
                        \frac
                            {\fee \ammR[0] \ammR[1] x_1 ((\ammR[0] + \fee x_0 + \fee x_1) - x_0 (\fee - 1))}
                            {(\ammR[0] + \fee x_0 + \fee x_1) (\ammR[0] + \fee x_0) ( \ammR[0] + x_0 + \fee x_1)}
                    \\
                    & = 
                        \frac
                            {\fee \ammR[0] \ammR[1] x_1 (\ammR[0] + x_0 + \fee x_1)}
                            {(\ammR[0] + \fee x_0 + \fee x_1) (\ammR[0] + \fee x_0) ( \ammR[0] + x_0 + \fee x_1)}
                    \\
                    & = 
                        \frac
                            {\fee \ammR[0] \ammR[1] x_1}
                            {(\ammR[0] + \fee x_0 + \fee x_1) (\ammR[0] + \fee x_0)}
                    \numberthis 
                    \label{eq:x1z}
                \\
                \\
                    \X[{\confGi}]{\tokT[0], \tokT[1]}
                    & = \lim_{z \rightarrow 0} \SX{z,\ammR[0] + x_0,\ammR[1] -\valSXa x_0}
                    \\
                    & = \frac
                        {\fee (\ammR[1] - \valSXa x_0)}
                        {\ammR[0] + x_0}
                      = \frac
                        {\fee \Big (\frac
                                {\ammR[1] (\ammR[0] + \fee x_0) - \fee \ammR[1] x_0}
                                {\ammR[0] + \fee x_0}
                            \Big )}
                        {\ammR[0] + x_0}
                      = \frac
                            {\fee \ammR[0] \ammR[1]}
                            {(\ammR[0] + \fee x_0) (\ammR[0] + x_0)}
                \end{align*}

                With these values in mind, we can continue the proof: 
                \begin{align*}
                    & \gain[{\confGi}]{\pmvA}{\txT(x_1)} + \epsilon_{\fee} < 0
                    \\
                    & \iff
                    x_1 (\valSXb \exchO{\tokT[1]} - \exchO{\tokT[0]}) + \exchO{\tokT[1]}(z - 1) (\valSXa x_0 + \valSXb x_1) < 0
                    \\
                    & \iff 
                    x_1 \valSXb \exchO{\tokT[1]} - x_1 \exchO{\tokT[0]} + \exchO{\tokT[1]}(z - 1) (\valSXa x_0 + \valSXb x_1) < 0
                    \\
                    & \iff 
                    \exchO{\tokT[1]} (x_1 \valSXb + (z - 1) (\valSXa x_0 + \valSXb x_1)) < x_1 \exchO{\tokT[0]}
                    \\ 
                    & \iff
                    \frac
                        {x_1 \valSXb + (z - 1) (\valSXa x_0 + \valSXb x_1)}
                        {x_1}
                    < \frac{\exchO{\tokT[0]}}{\exchO{\tokT[1]}}
                    && \text{by } x_1 > 0 \land \exchO{\tokT[1]} > 0
                    \\ 
                    & \iff
                    \frac
                        {x_1 \valSXb + (z - 1) (\valSXa x_0 + \valSXb x_1)}
                        {x_1}
                    < \X[\confGi]{\tokT[0], \tokT[1]}
                    \\
                    & \iff 
                    \frac
                        {\frac
                            {\fee \ammR[0] \ammR[1] x_1}
                            {(\ammR[0] + \fee x_0 + \fee x_1) (\ammR[0] + \fee x_0)}}
                        {x_1}
                    < \X[\confGi]{\tokT[0], \tokT[1]}
                    \\
                    & \iff 
                    \frac
                        {\fee \ammR[0] \ammR[1]}
                        {(\ammR[0] + \fee x_0 + \fee x_1) (\ammR[0] + \fee x_0)}
                    < \frac
                            {\fee \ammR[0] \ammR[1]}
                            {(\ammR[0] + \fee x_0) (\ammR[0] + x_0)}
                    \\
                    & \iff
                    \frac
                        {\fee \ammR[0] \ammR[1] (\ammR[0] + x_0) - \fee \ammR[0] \ammR[1] (\ammR[0] + \fee x_0 + \fee x_1)}
                        {(\ammR[0] + \fee x_0 + \fee x_1) (\ammR[0] + \fee x_0) (\ammR[0] + x_0)} < 0
                    \\
                    & \iff
                        \fee \ammR[0] \ammR[1] (\ammR[0] + x_0) - \fee \ammR[0] \ammR[1] (\ammR[0] + \fee x_0 + \fee x_1) < 0  && \text{by Den. } > 0
                    \\
                    & \iff
                        \fee \ammR[0] \ammR[1] (\ammR[0] + x_0 - \ammR[0] - \fee x_0 - \fee x_1) < 0
                    \\
                    & \iff
                        x_0 - \fee x_0 - \fee x_1 < 0    && \text{by } \fee \ammR[0] \ammR[1] < 0
                    \\
                    & \iff
                        \fee (x_0 + x_1) - x_0 > 0
                    \\
                    & \iff
                        \fee > \frac{x_0}{x_0 + x_1}
                    \\
                    & \iff
                    x_1 > \frac{x_0(1 - \fee)}{\fee}
                \end{align*}

                So, this is only true when: 
                \begin{align*}
                    & \iff
                    x_1 > \frac{x_0(1 - \fee)}{\fee}
                    \\
                    & \iff
                    x - x_0 > \frac{x_0(1 - \fee)}{\fee}
                    \\
                    & \iff
                    x > \frac{x_0(1 - \fee) + \fee x_0}{\fee}
                    \\
                    & \iff
                    x > \frac{x_0}{\fee}
                \end{align*}
            \end{proof}
    \end{itemize}
\end{proofof}

\albnote{collegato al mio commento di sopra riguardo la notazione T(x), notare che sotto non viene usata. Meglio uniformare. }

\albnote{Se ci focalizziamo soltanto nella constant product, potremo considerare introdurre il lemma di sotto prima dei lemmi/teoremi che ipotizzano (senza svelare) l'esistenza e unicità della $x0$ che porta l'amm in equilibrio. Ad esempio, Th 2.10. Ma anche 2.9.}

\begin{lem}[Balance swap value]
  \label{lem:arbitrage:balance}
  Let $\confG = \walA{\tokBal} \mid \amm{\ammR[0]:\tokT[0]}{\ammR[1]:\tokT[1]}$,
  and let:
  \begin{equation}
    \label{eq:arbitrage:balance}
    x_0 
    \; = \;
    \frac
        {-\sqrt{\exchO{\tokT[0]}} \ammR[0] (1 + \fee) + \sqrt{\ammR[0]} \sqrt{\exchO{\tokT[0]} \ammR[0] (-1 + \fee)^2 + 4 \exchO{\tokT[1]} \ammR[1] \fee^2}}
        {2 \sqrt{\exchO{\tokT[0]}} \fee}
  \end{equation}
  If $\SX{}$ is the constant product swap rate and $x_0 > 0$,
  then $\actAmmSwapExact{\pmvA}{}{x_0}{\tokT[0]}{\tokT[1]}$ such that $\confG \xrightarrow{\actAmmSwapExact{\pmvA}{}{x_0}{\tokT[0]}{\tokT[1]}} \confGi$
  brings the AMM to the balance state, i.e. 
  \begin{align*}
      \X{\tokT[0], \tokT[1]} = \X[\confGi]{\tokT[0], \tokT[1]}
  \end{align*}
\end{lem}

\alb{Sempre assumendo che ci concentriamo soltnato sul constant product, sotto farei riferimento allo x0 identificato da Lemma 2.11.}

\begin{proofof}{Theorem}{lem:arbitrage:balance}
    Before starting the proof, we rewrite some terms: 
    \begin{align*}
        x_0 + \ammR[0] & = 
        \frac
        {-\sqrt{\exchO{\tokT[0]}} \ammR[0] (1 + \fee) + \sqrt{\ammR[0]} \sqrt{\exchO{\tokT[0]} \ammR[0] (-1 + \fee)^2 + 4 \exchO{\tokT[1]} \ammR[1] \fee^2}}
        {2 \sqrt{\exchO{\tokT[0]}} \fee} + \ammR[0]
        \\
        & = 
        \frac
        {-\sqrt{\exchO{\tokT[0]}} \ammR[0] (1 + \fee) + \sqrt{\ammR[0]} \sqrt{\exchO{\tokT[0]} \ammR[0] (-1 + \fee)^2 + 4 \exchO{\tokT[1]} \ammR[1] \fee^2} + 2 \sqrt{\exchO{\tokT[0]}} \fee \ammR[0]}
        {2 \sqrt{\exchO{\tokT[0]}} \fee}
        \\
        & = 
        \frac
        {-\sqrt{\exchO{\tokT[0]}} \ammR[0] (1 - \fee) + \sqrt{\ammR[0]} \sqrt{\exchO{\tokT[0]} \ammR[0] (-1 + \fee)^2 + 4 \exchO{\tokT[1]} \ammR[1] \fee^2}}
        {2 \sqrt{\exchO{\tokT[0]}} \fee}
        \\
        & = 
        \frac
        {\sqrt{\exchO{\tokT[0]}} \ammR[0] (\fee - 1) + \sqrt{\ammR[0]} \sqrt{\exchO{\tokT[0]} \ammR[0] (-1 + \fee)^2 + 4 \exchO{\tokT[1]} \ammR[1] \fee^2}} 
        {2 \sqrt{\exchO{\tokT[0]}} \fee}
        \numberthis \label{eq:x0-r0}
        \\
        \fee x_0 + \ammR[0] & = \fee \cdot
        \frac
        {-\sqrt{\exchO{\tokT[0]}} \ammR[0] (1 + \fee) + \sqrt{\ammR[0]} \sqrt{\exchO{\tokT[0]} \ammR[0] (-1 + \fee)^2 + 4 \exchO{\tokT[1]} \ammR[1] \fee^2}}
        {2 \sqrt{\exchO{\tokT[0]}} \fee} + \ammR[0]
        \\
        & = \frac
        {-\sqrt{\exchO{\tokT[0]}} \ammR[0] (1 + \fee) + \sqrt{\ammR[0]} \sqrt{\exchO{\tokT[0]} \ammR[0] (-1 + \fee)^2 + 4 \exchO{\tokT[1]} \ammR[1] \fee^2} + 2 \sqrt{\exchO{\tokT[0]}} \ammR[0]}
        {2 \sqrt{\exchO{\tokT[0]}}}
        \\
        & = 
        \frac
        {-\sqrt{\exchO{\tokT[0]}} \ammR[0] (\fee - 1) + \sqrt{\ammR[0]} \sqrt{\exchO{\tokT[0]} \ammR[0] (-1 + \fee)^2 + 4 \exchO{\tokT[1]} \ammR[1] \fee^2}}
        {2 \sqrt{\exchO{\tokT[0]}}} \numberthis \label{eq:fee-x0-r0}
    \end{align*}
    And let 
    \begin{align*}
        m & = \sqrt{\ammR[0]} \sqrt{\exchO{\tokT[0]} \ammR[0] (-1 + \fee)^2 + 4 \exchO{\tokT[1]} \ammR[1] \fee^2}
        \\
        \valSXa & = \SX{x_0, \ammR[0], \ammR[1]} = \frac{\fee \ammR[1]}{\ammR[0] + \fee x_0}
    \end{align*}
    Then, 
    \begin{align*}
        \X[\confGi]{\tokT[0], \tokT[1]} & = 
        \lim_{z \rightarrow 0} \SX{z,\ammR[0] + x_0,\ammR[1] -\valSXa x_0}  && (\text{By Def.})
        \\
        & = \frac
            {\fee (\ammR[1] - x_0 \cdot \frac
                {\fee \ammR[1]}
                {\ammR[0] + \fee x_0})}
            {\ammR[0] + x_0}
        \\
        & = \frac
            {\fee (\frac
                {\ammR[0]\ammR[1] + \fee \ammR[1]x_0 - \fee \ammR[1] x_0}
                {\ammR[0] + \fee x_0})}
            {\ammR[0] + x_0}
        \\
        & = \frac
            {\fee \ammR[0] \ammR[1]}
            {(\ammR[0] + x_0) (\ammR[0] + \fee x_0)}
        \\
        & = \frac
            {\fee \ammR[0] \ammR[1]}
            {
            (\frac
            {\sqrt{\exchO{\tokT[0]}} \ammR[0] (\fee - 1) + m}
            {2 \sqrt{\exchO{\tokT[0]}} \fee})
            (\frac
            {-\sqrt{\exchO{\tokT[0]}} \ammR[0] (\fee - 1) + m}
            {2 \sqrt{\exchO{\tokT[0]}}})}   && (\text{By Subst.})
        \\
        & = \frac
            {\fee \ammR[0] \ammR[1]}
            {\frac
                {(\sqrt{\exchO{\tokT[0]}} \ammR[0] (\fee - 1) + m)(-\sqrt{\exchO{\tokT[0]}} \ammR[0] (\fee - 1) + m)}
                {4 \exchO{\tokT[0]} \fee}}
        \\
        & = \frac
            {\fee \ammR[0] \ammR[1] (4 \exchO{\tokT[0]} \fee)}
            {m^2 - (\sqrt{\exchO{\tokT[0]}} \ammR[0] (\fee - 1))^2}
        \\
        & = \frac
            {\fee \ammR[0] \ammR[1] (4 \exchO{\tokT[0]} \fee)}
            {\ammR[0] (\exchO{\tokT[0]} \ammR[0] (\fee - 1)^2 + 4 \exchO{\tokT[1]} \ammR[1] \fee ^2) - \exchO{\tokT[0]} \ammR[0]^2 (\fee - 1)^2}
        \\
        & = \frac
            {\fee \ammR[0] \ammR[1] (4 \exchO{\tokT[0]} \fee)}
            {\exchO{\tokT[0]} \ammR[0]^2 (\fee - 1)^2 + 4 \exchO{\tokT[1]} \ammR[0] \ammR[1] \fee ^2 - \exchO{\tokT[0]} \ammR[0]^2 (\fee - 1)^2}
        \\
        & = \frac
            {\fee \ammR[0] \ammR[1] (4 \exchO{\tokT[0]} \fee)}
            { 4 \exchO{\tokT[1]} \ammR[0] \ammR[1] \fee ^2 }
        \\ & = 
        \frac{\exchO{\tokT[0]}}{\exchO{\tokT[1]}}
    \end{align*}
\end{proofof}

\begin{thm}[Max Gain Value]
  \label{thm:fee:max-gain}  
  Let $\confG = \walA{\tokBal} \mid \amm{\ammR[0]:\tokT[0]}{\ammR[1]:\tokT[1]} \mid \confD$ 
  be such that $\tokBal{\tokM{\tokT[0]}{\tokT[1]}} = 0$.
  For all $x > 0$, 
  let $\txT(x) = \actAmmSwapExact{\pmvA}{}{x}{\tokT[0]}{\tokT[1]}$.
  Let $x_0$ be such that:
  \begin{equation}
    \label{eq:fee:arbitrage:max:x0}
    % \SX{0,\ammR[0]+x_0,\ammR[1]-x_0 \cdot \SX{x_0,\ammR[0],\ammR[1]}}
    \X[\confGi]{\tokT[0],\tokT[1]} = \X{\tokT[0],\tokT[1]}
    \qquad
    \text{ where }
    \confG \xrightarrow{\txT(x_0)} \confGi
  \end{equation}
  Let $x_{max}$ be \albnote{rimuovere ``such that''-- a me sembra $x_{max}$ sia ben definita (non c'è scelta)} such that: 
  \begin{equation}
    \label{eq:fee:arbitrage:max-value}
    x_{max} = x_0 + \frac
        {-\sqrt{\exchO{\tokT[0]}} \ammR[0] - \sqrt{\exchO{\tokT[0]}} \fee x_0 + \sqrt{\exchO{\tokT[1]} \fee \ammR[0] \ammR[1]}}
        {\sqrt{\exchO{\tokT[0]}} \fee}
  \end{equation}
  In particular, $x_0 < x_{max} < \frac{x_0}{\fee}$.
  If $\SX{}$ is the constant product swap rate function, then:
  \albnote{ma non vale semplicemente per tutte le $x$ diverse da $x_{max}$?}
  \[
    \forall x \neq x_{max}
    \; : \;
    x_0 < x < \frac{x_0}{\fee} \implies
    \gain[{\confG}]{\pmvA}{\txT(x_{max})}
    \; > \;
    \gain[{\confG}]{\pmvA}{\txT(x)}
  \]
\end{thm}

\begin{proofof}{Theorem}{thm:fee:max-gain} 

    Let $x_1 > 0$ be such that $x_{max} = x_0 + x_1$. Then, as we stated in the proof of Theorem ~\ref{thm:fee:arbitrage}, by Lemma ~\ref{lem:fee:swap-gain:additive} we know that: 

        \[
        \gain[\confG]{\pmvA}{\txT(x_{max})} 
        \; = \;
        \gain[\confG]{\pmvA}{\txT(x_0)} + \gain[{\confGi}]{\pmvA}{\txT(x_1)} + \epsilon_{\fee}
        \]
    
    We have proven in Theorem ~\ref{thm:fee:arbitrage} that for $x < \frac{x_0}{\fee}$, $\gain[{\confGi}]{\pmvA}{\txT(x_1)} + \epsilon_{\fee} > 0$. Since $x_0$ is fixed, we want to find the maximum of the function: 
    \begin{equation}
        f(x_1) = \gain[\confG]{\pmvA}{\txT(x_0)} + \gain[{\confGi}]{\pmvA}{\txT(x_1)} + \epsilon_{\fee}
    \end{equation}
    where $0 < x_1 < \frac{x_0(1 - \fee)}{\fee}$

    To find the maximum, first we look at the function values at the extremes of its domain(for simplicity, let  $b = \frac{x_0(1 - \fee)}{\fee}$): 

    \begin{align*}
        \lim_{x_1 \rightarrow 0} f(x_1) & = 
        \lim_{x_1 \rightarrow 0} \gain[\confG]{\pmvA}{\txT(x_0)} + \gain[{\confGi}]{\pmvA}{\txT(x_1)} + \epsilon_{\fee}
        \\
        & = \lim_{x_1 \rightarrow 0} \gain[\confG]{\pmvA}{\txT(x_0)} +
        x_1 \valSXb \exchO{\tokT[1]} - x_1 \exchO{\tokT[0]} + \exchO{\tokT[1]} (z - 1)(\valSXa x_0 + \valSXb x_1)
        \\
        & = \lim_{x_1 \rightarrow 0} \gain[\confG]{\pmvA}{\txT(x_0)}
        - x_1 \exchO{\tokT[0]} + \exchO{\tokT[1]} ((z-1)(\valSXa x_0 + \valSXb x_1) + \valSXb x_1)
        \\
        & = \lim_{x_1 \rightarrow 0} \gain[\confG]{\pmvA}{\txT(x_0)}
            - x_1 \exchO{\tokT[0]} + 
            \frac
                {\exchO{\tokT[1]} \fee \ammR[0] \ammR[1] x_1}
                {(\ammR[0] + \fee x_0) (\ammR[0] + \fee x_0 + \fee x_1)}
        && (\text{By~\ref{eq:x1z}})
        \\
        & = \gain[\confG]{\pmvA}{\txT(x_0)} - 0 + \frac{0}{(\ammR[0] + \fee x_0) (\ammR[0] + \fee x_0)}
        \\
        & = \gain[\confG]{\pmvA}{\txT(x_0)}
        \\
        \phantom{space} \\
        \lim_{x_1 \rightarrow b} f(x_1) & = 
        \lim_{x_1 \rightarrow b} \gain[\confG]{\pmvA}{\txT(x_0)} + \gain[{\confGi}]{\pmvA}{\txT(x_1)} + \epsilon_{\fee}
        \\
        & = \lim_{x_1 \rightarrow b} \gain[\confG]{\pmvA}{\txT(x_0)}
            - x_1 \exchO{\tokT[0]} + 
            \frac
                {\exchO{\tokT[1]} \fee \ammR[0] \ammR[1] x_1}
                {(\ammR[0] + \fee x_0) (\ammR[0] + \fee x_0 + \fee x_1)}
        \\
        & = \gain[\confG]{\pmvA}{\txT(x_0)}
            - \frac{x_0(1 - \fee)}{\fee} \exchO{\tokT[0]} + 
            \frac
                {\exchO{\tokT[1]} \fee \ammR[0] \ammR[1] \frac{x_0(1 - \fee)}{\fee}}
                {(\ammR[0] + \fee x_0) (\ammR[0] + \fee x_0 + \fee \frac{x_0(1 - \fee)}{\fee})}
        \\
        & = \gain[\confG]{\pmvA}{\txT(x_0)}
            - \frac{x_0(1 - \fee)}{\fee} \exchO{\tokT[0]} + 
            \frac
                {\exchO{\tokT[1]} \ammR[0] \ammR[1] x_0(1 - \fee)}
                {(\ammR[0] + \fee x_0) (\ammR[0] + \fee x_0 + x_0(1 - \fee))}
        \\
        & = \gain[\confG]{\pmvA}{\txT(x_0)}
            - \frac{x_0(1 - \fee)}{\fee} \exchO{\tokT[0]} + 
            \frac
                {\exchO{\tokT[1]} \ammR[0] \ammR[1] x_0(1 - \fee)}
                {(\ammR[0] + \fee x_0) (\ammR[0] + x_0)}
        \\
        & = \gain[\confG]{\pmvA}{\txT(x_0)} +
            \frac
                {- x_0(1 - \fee)(\ammR[0] + \fee x_0) (\ammR[0] + x_0) \exchO{\tokT[0]} + \exchO{\tokT[1]} \fee \ammR[0] \ammR[1] x_0(1 - \fee)}
                {\fee (\ammR[0] + \fee x_0) (\ammR[0] + x_0)}
        \\
        & = \gain[\confG]{\pmvA}{\txT(x_0)} +
            \frac
                {- x_0(1 - \fee) \Big ( \exchO{\tokT[1]} \fee \ammR[0] \ammR[1] - \exchO{\tokT[0]}(\ammR[0] + \fee x_0) (\ammR[0] + x_0) \Big )}
                {\fee (\ammR[0] + \fee x_0) (\ammR[0] + x_0)}
        \\
        & = \gain[\confG]{\pmvA}{\txT(x_0)} +
            \frac
                {- x_0(1 - \fee) \Big ( \exchO{\tokT[1]} \fee \ammR[0] \ammR[1] - \exchO{\tokT[0]} \frac{4 \exchO{\tokT[1]} \ammR[0] \ammR[1] \fee^2}{4 \exchO{\tokT[0]} \fee} \Big )}
                {\fee (\ammR[0] + \fee x_0) (\ammR[0] + x_0)}
        && (\text{By~\ref{eq:x0-r0},~\ref{eq:fee-x0-r0}})
        \\
        & = \gain[\confG]{\pmvA}{\txT(x_0)} +
            \frac
                {- x_0(1 - \fee) \Big ( \exchO{\tokT[1]} \fee \ammR[0] \ammR[1] - \exchO{\tokT[1]} \fee \ammR[0] \ammR[1] \Big )}
                {\fee (\ammR[0] + \fee x_0) (\ammR[0] + x_0)}
        \\
        & = \gain[\confG]{\pmvA}{\txT(x_0)} +
            \frac
                {- x_0(1 - \fee) \cdot 0}
                {\fee (\ammR[0] + \fee x_0) (\ammR[0] + x_0)}
        \\
        & = \gain[\confG]{\pmvA}{\txT(x_0)}
    \end{align*}

    Now, we have to study the critical points of the function. To do that, we first have to compute the first derivative of $f(x_1)$:
    \begin{align*}
        f'(x_1) = &  \Big [
             - x_1 \exchO{\tokT[0]} + 
            \frac
                {\exchO{\tokT[1]} \fee \ammR[0] \ammR[1] x_1}
                {(\ammR[0] + \fee x_0) (\ammR[0] + \fee x_0 + \fee x_1)}
                \Big ]'
        \\
        = &  -\exchO{\tokT[0]} + \Big [ 
            \frac
                {\exchO{\tokT[1]} \fee \ammR[0] \ammR[1] x_1}
                {(\ammR[0] + \fee x_0) (\ammR[0] + \fee x_0 + \fee x_1)}
                \Big ]'
        \\
        = &  -\exchO{\tokT[0]} +
            \frac
                {[\exchO{\tokT[1]} \fee \ammR[0] \ammR[1] x_1]' (\ammR[0] + \fee x_0) (\ammR[0] + \fee x_0 + \fee x_1)}
                {((\ammR[0] + \fee x_0) (\ammR[0] + \fee x_0 + \fee x_1))^2}
                \\
            & -  \frac
                {(\exchO{\tokT[1]} \fee \ammR[0] \ammR[1] x_1)[(\ammR[0] + \fee x_0) (\ammR[0] + \fee x_0 + \fee x_1)]'}
                {((\ammR[0] + \fee x_0) (\ammR[0] + \fee x_0 + \fee x_1))^2}
        \\
        = &  -\exchO{\tokT[0]} +
            \frac
                {\exchO{\tokT[1]} \fee \ammR[0] \ammR[1] (\ammR[0] + \fee x_0) (\ammR[0] + \fee x_0 + \fee x_1)}
                {((\ammR[0] + \fee x_0) (\ammR[0] + \fee x_0 + \fee x_1))^2}
                \\
            & -  \frac
                {(\exchO{\tokT[1]} \fee \ammR[0] \ammR[1] x_1) \fee (\ammR[0] + \fee x_0)}
                {((\ammR[0] + \fee x_0) (\ammR[0] + \fee x_0 + \fee x_1))^2}
        \\
        = &  -\exchO{\tokT[0]} +
            \frac
                {\exchO{\tokT[1]} \fee \ammR[0] \ammR[1] \Big ( (\ammR[0] + \fee x_0) (\ammR[0] + \fee x_0 + \fee x_1) - \fee x_1 (\ammR[0] + \fee x_0) \Big )}
                {(\ammR[0] + \fee x_0)^2 (\ammR[0] + \fee x_0 + \fee x_1)^2}
        \\
        = &  -\exchO{\tokT[0]} +
            \frac
                {\exchO{\tokT[1]} \fee \ammR[0] \ammR[1] (\ammR[0] + \fee x_0) \Big ( \ammR[0] + \fee x_0 + \fee x_1 - \fee x_1 \Big )}
                {(\ammR[0] + \fee x_0)^2 (\ammR[0] + \fee x_0 + \fee x_1)^2}
        \\
        = &  -\exchO{\tokT[0]} +
            \frac
                {\exchO{\tokT[1]} \fee \ammR[0] \ammR[1] (\ammR[0] + \fee x_0) ( \ammR[0] + \fee x_0)}
                {(\ammR[0] + \fee x_0)^2 (\ammR[0] + \fee x_0 + \fee x_1)^2}
        \\
        = &  -\exchO{\tokT[0]} +
            \frac
                {\exchO{\tokT[1]} \fee \ammR[0] \ammR[1]}
                {(\ammR[0] + \fee x_0 + \fee x_1)^2}
    \end{align*}

    Now, we want to prove that 
    \begin{equation}
        \label{eq:partial-xmax}
        x' =
        \frac
            {-\sqrt{\exchO{\tokT[0]}} \ammR[0] - \sqrt{\exchO{\tokT[0]}} \fee x_0 + \sqrt{\exchO{\tokT[1]} \fee \ammR[0] \ammR[1]}}
            {\sqrt{\exchO{\tokT[0]}} \fee}
    \end{equation}
    is a critical point of $f(x_1)$, i.e. $f'(x') = 0$. 
    First, let's rewrite: 
    \begin{align*}
        \fee x' = & 
        \fee \cdot \frac
            {-\sqrt{\exchO{\tokT[0]}} \ammR[0] - \sqrt{\exchO{\tokT[0]}} \fee x_0 + \sqrt{\exchO{\tokT[1]} \fee \ammR[0] \ammR[1]}}
            {\sqrt{\exchO{\tokT[0]}} \fee}
        \\
        = & 
        \frac
            {-\sqrt{\exchO{\tokT[0]}} \ammR[0] - \sqrt{\exchO{\tokT[0]}} \fee x_0 + \sqrt{\exchO{\tokT[1]} \fee \ammR[0] \ammR[1]}}
            {\sqrt{\exchO{\tokT[0]}}}
        \\
        = & 
        \frac
            {-\sqrt{\exchO{\tokT[0]}} (\ammR[0] + \fee x_0)}
            {\sqrt{\exchO{\tokT[0]}}} + 
        \frac
            {\sqrt{\exchO{\tokT[1]} \fee \ammR[0] \ammR[1]}}
            {\sqrt{\exchO{\tokT[0]}}}
        \\
        = & 
        - (\ammR[0] + \fee x_0)
            + 
        \frac
            {\sqrt{\exchO{\tokT[1]} \fee \ammR[0] \ammR[1]}}
            {\sqrt{\exchO{\tokT[0]}}} \numberthis \label{eq:feex'}
    \end{align*}

    Finally, 

    \begin{align*}
        f'(x') = & 
        -\exchO{\tokT[0]} +
            \frac
                {\exchO{\tokT[1]} \fee \ammR[0] \ammR[1]}
                {(\ammR[0] + \fee x_0 + \fee x')^2}
        \\
        = & 
        -\exchO{\tokT[0]} +
            \frac
                {\exchO{\tokT[1]} \fee \ammR[0] \ammR[1]}
                {(\ammR[0] + \fee x_0 - (\ammR[0] + \fee x_0)
                    + 
                    \frac
                    {\sqrt{\exchO{\tokT[1]} \fee \ammR[0] \ammR[1]}}
                    {\sqrt{\exchO{\tokT[0]}}})^2}
        \\
        = & 
        -\exchO{\tokT[0]} +
            \frac
                {\exchO{\tokT[1]} \fee \ammR[0] \ammR[1]}
                {(\frac
                    {\sqrt{\exchO{\tokT[1]} \fee \ammR[0] \ammR[1]}}
                    {\sqrt{\exchO{\tokT[0]}}})^2}
        \\
        = & 
        -\exchO{\tokT[0]} +
            \frac
                {\exchO{\tokT[1]} \fee \ammR[0] \ammR[1]}
                {\frac
                    {\exchO{\tokT[1]} \fee \ammR[0] \ammR[1]}
                    {\exchO{\tokT[0]}}}
        \\
        = & 
        -\exchO{\tokT[0]} + \exchO{\tokT[0]} = 0
    \end{align*}
    
    
\end{proofof}