\chapter{Proofs}


\begin{proofof}{Lemma}{lem:fee:swap:gain}
    Proof is the same as in~\cite{BCL22lmcs}.
\end{proofof}

\begin{proofof}{Lemma}{lem:fee:swap:gain-SX-X-fee}
    Proof is the same as in~\cite{BCL22lmcs}.
\end{proofof}

\begin{proofof}{Lemma}{lem:fee:additive:zeta-reduced}
    \begin{align*}
        \Z x y {r_0} {r_1} & =
        \frac
            {\Big ( (\fee \ammR[1] x)(\ammR[0] + \fee x + \fee y) + (\fee \ammR[1] \ammR[0] y)\Big ) \cdot (\ammR[0] + x + \fee y)}
            {(\ammR[0] + \fee x + \fee y) \cdot \Big ( (\fee \ammR[1] x)(\ammR[0] + x + \fee y) + (\fee \ammR[1] \ammR[0] y)\Big )}
        \\
        & = 
        \frac
            {\Big ( \fee \ammR[0] \ammR[1] x + \fee ^ 2 \ammR[1] x^2 + \fee^2 \ammR[1] x y + \fee \ammR[0] \ammR[1] y\Big ) \cdot (\ammR[0] + x + \fee y)}
            {(\ammR[0] + \fee x + \fee y) \cdot \Big ( \fee \ammR[0] \ammR[1] x + \fee \ammR[1] x^2 + \fee^2 \ammR[1] x y + \fee \ammR[0] \ammR[1] y \Big )}
        \\
        & = 
        \frac
            {\Big ( \fee \ammR[1] x (\ammR[0] + \fee x) + \fee \ammR[1] y (\ammR[0] + \fee x)\Big ) \cdot (\ammR[0] + x + \fee y)}
            {(\ammR[0] + \fee x + \fee y) \cdot \Big ( \fee \ammR[1] (\ammR[0] x + x^2 + \fee x y + \ammR[0] y) \Big )}
        \\
        & = 
        \frac
            {\fee \ammR[1] \Big ( x (\ammR[0] + \fee x) + y (\ammR[0] + \fee x)\Big ) \cdot (\ammR[0] + x + \fee y)}
            {\fee \ammR[1] (\ammR[0] + \fee x + \fee y) (\ammR[0] x + x^2 + \fee x y + \ammR[0] y)}
        \\
        & = 
        \frac
        {(x + y)(\ammR[0] + \fee x)(\ammR[0] + x + \fee y)}
        {(\ammR[0] + \fee x + \fee y)(x^2 + \ammR[0] x +\fee x y + \ammR[0] y)} 
    \end{align*}
\end{proofof}

\begin{proofof}{Lemma}{lem:fee:additive:zeta-gt-one}
    Assume that $\phi < 1$, then: 
    \begin{align*}
        & \Z x y {r_0} {r_1} > 1 
        \\
        \iff & 
        \Z x y {r_0} {r_1} - 1 > 0
        \\
        \iff & 
        \frac
        {(x + y)(\ammR[0] + \fee x)(\ammR[0] + x + \fee y)}
        {(\ammR[0] + \fee x + \fee y)(x^2 + \ammR[0] x +\fee x y + \ammR[0] y)} 
        - 1 > 0 && \quad \text{By~\ref{lem:fee:additive:zeta-reduced}}
        \\
        \iff & 
        (x + y)(\ammR[0] + \fee x)(\ammR[0] + x + \fee y) 
        \\ & 
        - (\ammR[0] + \fee x + \fee y)(x^2 + \ammR[0] x +\fee x y + \ammR[0] y) > 0
        && \quad  \text{Denum > 0}
        \\
        \iff & 
        \ammR[0] x y + x(\ammR[0] + \fee x)(\ammR[0] + x + \fee y) + \ammR[0] y (\ammR[0] + \fee y) + \fee x y(\ammR[0] + x + \fee y) 
        \\ & 
        - \fee \ammR[0] x y - \fee x (x^2 + \ammR[0] x + \fee x y) - (\ammR[0] + \fee y) (x^2 + \ammR[0] x + \fee x y + \ammR[0] y)
        \\ & > 0
        \\
        \iff & 
        \ammR[0] x y (1 - \fee) + (\ammR[0] + \fee x)(x^2 + \ammR[0] x + \fee x y) + \fee y (x^2 + \ammR[0] x + \fee x y)
        \\ & 
        + \ammR[0] y (\ammR[0] + \fee y) - \fee x (x^2 + \ammR[0] x + \fee x y) - \ammR[0] y (\ammR[0] + \fee y) 
        \\ & 
        - (\ammR[0] + \fee y)(x^2 + \ammR[0] x + \fee x y) > 0
        \\ 
        \iff &
        \ammR[0] x y (1 - \fee) + (\ammR[0] + \fee x)(x^2 + \ammR[0] x + \fee x y) + \fee y (x^2 + \ammR[0] x + \fee x y)
        \\  & 
        - \fee x (x^2 + \ammR[0] x + \fee x y) - (\ammR[0] + \fee y)(x^2 + \ammR[0] x + \fee x y) > 0
        \\
        \iff & 
        \ammR[0] x y (1 - \fee) + (x^2 + \ammR[0] x + \fee x y)(\ammR[0] + \fee x + \fee y - \fee x - \ammR[0] - \fee y) > 0
        \\
        \iff & 
        \ammR[0] x y (1 - \fee) > 0
        \\ 
        \iff & (1 - \fee) > 0
        \\
        \iff & 
        \fee < 1
    \end{align*}
\end{proofof}

\begin{proofof}{Lemma}{lem:fee:swap-gain:additive}
By Definition~\ref{defi:fee:sr-additivity} we have that
    \begin{align*}
        \valSXc = \SX{x_0+x_1, \ammR[0], \ammR[1]} = 
        \frac{\valSXa x_0 + \valSXa x_1}{x_0 + x_1} \cdot z
    \end{align*}
    So, 
    \begin{align*}
        & \gain[\confG]{\pmvA}{\txT(x_0 + x_1)} - \gain[\confG]{\pmvA}{\txT(x_0)}
        \\
        & = \valSXc (x_0 + x_1)\exchO{\tokT[1]} - (x_0 + x_1) \exchO{\tokT[0]} 
          - \valSXa x_0 \exchO{\tokT[1]} + x_0 \exchO{\tokT[0]}
        \\
        & = (\valSXc (x_0 + x_1) - \valSXa x_0) \exchO{\tokT[1]} - x_1 \exchO{\tokT[0]}
        \\
        & = (z \valSXa x_0 + z \valSXb x_1 - \valSXa x_0) \exchO{\tokT[1]} - x_1 \exchO{\tokT[0]}
        \\
        & = z \valSXa x_0 \exchO{\tokT[1]} + z \valSXb x_1 \exchO{\tokT[1]} - \valSXa x_0 \exchO{\tokT[1]} - x_1 \exchO{\tokT[0]}
        \\
        & = z \valSXa x_0 \exchO{\tokT[1]} + z \valSXb x_1 \exchO{\tokT[1]} - \valSXa x_0 \exchO{\tokT[1]} - x_1 \exchO{\tokT[0]} + \valSXb x_1 \exchO{\tokT[1]} - \valSXb x_1 \exchO{\tokT[1]}
        \\
        & = (\valSXb x_1 \exchO{\tokT[1]} - x_1 \exchO{\tokT[0]}) + (z \valSXa x_0 \exchO{\tokT[1]} + z \valSXb x_1 \exchO{\tokT[1]} - \valSXa x_0 \exchO{\tokT[1]} - \valSXb x_1 \exchO{\tokT[0]})
        \\
        & = \gain[\confGi]{\pmvA}{\txT(x_1)} + \exchO{\tokT[1]}(z-1)(\valSXa x_0 + \valSXb x_1) && (\text{by Def.~\ref{lem:fee:swap:gain}})
        \\
        & = \gain[\confGi]{\pmvA}{\txT(x_1)} + \epsilon_{\fee} && (\text{by Def. of $\epsilon_{\fee}$})
    \end{align*}
\end{proofof}

\begin{proofof}{Lemma}{lem:fee:constprod:output-bound}
    \begin{align*}
        & x \cdot \SX{x,\ammR[0],\ammR[1]} < \ammR[1]
        \\
        \iff & 
        x \cdot \frac{\fee \ammR[1]}{\ammR[0] + \fee  x} < \ammR[1]
        \\
        \iff & 
        x \fee \ammR[1] < \ammR[1] + (\ammR[0] + \fee x)
        \\
        \iff & 
        \fee x \ammR[1] < \ammR[0] \ammR[1] + \fee x \ammR[1]
        \\
        \iff & 
        \ammR[0] \ammR[1] > 0
        \\
        \iff & True
    \end{align*}
\end{proofof}

\begin{proofof}{Lemma}{lem:fee:constprod:strict-mono}
    There are two possible cases either $x' < x \lor r_0' < r_0 \lor r_1 < r_1'$ or not. in both cases, it is always true that $x' \leq x \land r_0' \leq r_0 \land r_1 \leq r_1'$

    \begin{enumerate}
        \item $\neg (x' < x \lor r_0' < r_0 \lor r_1 < r_1')$ : \\
        In this case we want to prove: 
        \begin{align*}
            & \SX{x, r_0, r_1} \leq \SX{x', r_0', r_1'}
            \\ \iff & 
            \frac{\fee \ammR[1]}{\ammR[0] + \fee x} \leq \frac{\fee \ammR[1]'}{\ammR[0]' + \fee x'}
            \\ \iff & 
            \fee \ammR[1] (\ammR[0]' + \fee x') \leq \fee \ammR[1]' (\ammR[0] + \fee x)
        \end{align*}
        This is true iff 
        \begin{align*}
            \fee \ammR[1] \leq \fee \ammR[1]'\iff
            \ammR[1] \leq \ammR[1]' \quad \text{By assm.}
        \end{align*}
        and 
        \begin{align*}
            & \ammR[0]' + \fee x' \leq \ammR[0] + \fee x
            \\ \iff & \Big ( 
            (\ammR[0]' \leq \ammR[0]) \land (\fee x' \leq \fee x) \Big ) \quad && \text{By assm.}
            \\ \iff & x' \leq x \quad && \text{By assm.}
        \end{align*}

        \item $x' < x \lor r_0' < r_0 \lor r_1 < r_1'$: \\
        In this case we have three different cases: 
        \begin{enumerate}
            \item Case $x' < x$: \\
            \begin{align*}
                & \SX{x, r_0, r_1} < \SX{x', r_0', r_1'}
                \\ \iff & 
                \frac{\fee \ammR[1]}{\ammR[0] + \fee x} < \frac{\fee \ammR[1]'}{\ammR[0]' + \fee x'}
                \\ \iff & 
                \fee \ammR[1] (\ammR[0]' + \fee x') < \fee \ammR[1]' (\ammR[0] + \fee x)
            \end{align*}
            Since we know that $\fee \ammR[1] \leq \fee \ammR[1]'$ and $\ammR[0]' + \fee x' < \ammR[0] + \fee x$ by assumptions, then the proof is finished. 
            \item Case $\ammR[0]' < \ammR[0]$: \\
            \begin{align*}
                & \SX{x, r_0, r_1} < \SX{x', r_0', r_1'}
                \\ \iff & 
                \frac{\fee \ammR[1]}{\ammR[0] + \fee x} < \frac{\fee \ammR[1]'}{\ammR[0]' + \fee x'}
                \\ \iff & 
                \fee \ammR[1] (\ammR[0]' + \fee x') < \fee \ammR[1]' (\ammR[0] + \fee x)
            \end{align*}
            Since we know that $\fee \ammR[1] \leq \fee \ammR[1]'$ and $\ammR[0]' + \fee x' < \ammR[0] + \fee x$ by assumptions, then the proof is finished. 
            \item Case $\ammR[1] < \ammR[1]'$: \\
            \begin{align*}
                & \SX{x, r_0, r_1} < \SX{x', r_0', r_1'}
                \\ \iff & 
                \frac{\fee \ammR[1]}{\ammR[0] + \fee x} < \frac{\fee \ammR[1]'}{\ammR[0]' + \fee x'}
                \\ \iff & 
                \fee \ammR[1] (\ammR[0]' + \fee x') < \fee \ammR[1]' (\ammR[0] + \fee x)
            \end{align*}
            Since we know that $\fee \ammR[1] < \fee \ammR[1]'$ and $\ammR[0]' + \fee x' \leq \ammR[0] + \fee x$ by assumptions, then the proof is finished.
        \end{enumerate}
    \end{enumerate}
\end{proofof}


\begin{proofof}{Lemma}{lem:fee:constprod:additivity}
    Let 
    \begin{align*}
        \valSXa & = \SX{x, \ammR[0], \ammR[1]} = \frac{\fee \ammR[1]}
                                                    {\ammR[0] + \fee x}
        \\
        \valSXb & = \SX{y, \ammR[0] + x, \ammR[1] - \valSXa x} = 
        \frac
            {\fee (\ammR[1] - \valSXa x)}
            {\ammR[0] + x + \fee y}
        = \frac 
            {\fee \cdot \Big ( \ammR[1] - \frac{\fee \ammR[1] x}{\ammR[0] + \fee x}\Big )}
            {\ammR[0] + x + \fee y}    
         = \frac 
            {\fee \Big ( \frac{\ammR[1](\ammR[0] + \fee x) -   \fee \ammR[1] x}{\ammR[0] + \fee x}\Big )}
            {\ammR[0] + x + \fee y}
        \\
        & = \frac{\fee (\ammR[1] \ammR[0] + \fee \ammR[1] x - \fee \ammR[1] x)}
                 {(\ammR[0] + \fee x)(\ammR[0] + x + \fee y)}
          = \frac{\fee \ammR[1] \ammR[0]}
                 {(\ammR[0] + \fee x)(\ammR[0] + x + \fee y)} 
    \end{align*}

    Then, 
    \begin{align*}
        \frac{\valSXa x + \valSXb y}{x+y} \cdot z 
        & = 
        \frac{1}{x + y} \Big ( 
            \frac{\fee \ammR[1] x}{\ammR[0] + \fee x} + 
            \frac{\fee \ammR[1] \ammR[0] y}{(\ammR[0] + \fee x)(\ammR[0] + x +\fee y)}
        \Big ) \cdot z
        \\
        & = 
        \frac{1}{x + y} \Big (
            \frac
                {\fee \ammR[0] \ammR[1] x + \fee \ammR[1] x^2 + \fee^2 \ammR[1] x y + \fee \ammR[0] \ammR[1] y}
                {(\ammR[0] + \fee x)(\ammR[0] + x +\fee y)}
        \Big ) \cdot z
        \\
        & = 
        \frac{1}{x + y} \Big (
            \frac
                {(\fee \ammR[1] x)(\ammR[0] + x + \fee y) + (\fee \ammR[1] \ammR[0] y)}
                {(\ammR[0] + \fee x)(\ammR[0] + x +\fee y)}
        \Big ) \cdot 
        \\
        & \frac
            {\Big ( (\fee \ammR[1] x)(\ammR[0] + \fee x + \fee y) + (\fee \ammR[1] \ammR[0] y)\Big ) \cdot (\ammR[0] + x + \fee y)}
            {(\ammR[0] + \fee x + \fee y) \cdot \Big ( (\fee \ammR[1] x)(\ammR[0] + x + \fee y) + (\fee \ammR[1] \ammR[0] y)\Big )}
        \\
        & = 
        \frac{1}{x + y}
            \frac
                {(\fee \ammR[1] x)(\ammR[0] + \fee x + \fee y) + (\fee \ammR[1] \ammR[0] y)}
                {(\ammR[0] + \fee x)(\ammR[0] + \fee x +\fee y)}
        \\        
        & = 
        \frac{1}{x + y}
            \frac
                {\fee \ammR[0] \ammR[1] x + \fee^2 \ammR[1] x^2 + \fee^2 \ammR[1] x y + \fee \ammR[1] \ammR[0] y}
                {(\ammR[0] + \fee x)(\ammR[0] + \fee x +\fee y)}
        \\        
        & = 
        \frac{1}{x + y}
            \frac
                {\fee \ammR[1] (\ammR[0] x + \fee x^2 + \fee x y + \ammR[0] y)}
                {(\ammR[0] + \fee x)(\ammR[0] + \fee x +\fee y)}
        \\        
        & = 
        \frac{1}{x + y}
            \frac
                {\fee \ammR[1] (\ammR[0] + \fee x)(x + y)}
                {(\ammR[0] + \fee x)(\ammR[0] + \fee x +\fee y)}
        \\        
        & = 
        \frac{\fee \ammR[1]}{\ammR[0] + \fee (x + y)}
        \\
        & = 
        \SX{x+y, \ammR[0], \ammR[1]}
    \end{align*}
\end{proofof}

\begin{proofof}{Lemma}{lem:fee:sx-rate-vs-int-rate}
    We split the proof in two parts: 
    \begin{itemize}
        \item $\SX{x_0, \ammR[0], \ammR[1]} < \X[\Gamma]{\tokT[0],\tokT[1]}$: 
        \begin{align*}
            \X[\confG]{\tokT[0],\tokT[1]} 
                & = \lim_{z \rightarrow 0} \SX{z,\ammR[0],\ammR[1]} && \text{(by Def. of X)}
                \\
                & > \SX{x_0, \ammR[0], \ammR[1]}                    && \text{Strict. Mono.}
        \end{align*}

        \item $\X[\confGi]{\tokT[0],\tokT[1]} < \SX{x_0, \ammR[0], \ammR[1]}$:

            Let $\valSXa = \SX{x_0, \ammR[0], \ammR[1]}$. 
            
            Before continuing the proof, lets first prove that
                \begin{equation}
                    \label{eq:SX-less-SX-fee}
                    \frac{\fee(\ammR[1] - \valSXa x_0)}{\ammR[0] + x_0} < \frac{\fee \ammR[1]}{\ammR[0] + \fee x_0}
                \end{equation}
                \begin{proof}
                    \begin{align*}
                        & \frac{\fee(\ammR[1] - \valSXa x_0)}{\ammR[0] + x_0} < \frac{\fee 
                          \ammR[1]}{\ammR[0] + \fee x_0}
                        \\
                        \iff & 
                        \fee (\ammR[1] - \valSXa x_0)(\ammR[0] + \fee x_0) < \fee \ammR[1] (\ammR[0] + x_0)
                        \\
                        \iff & 
                        \fee \ammR[1](\ammR[0] + \fee x_0) - \fee \valSXa x_0 (\ammR[0] + \fee x_0) - \fee \ammR[1] (\ammR[0] + \fee x_0) < 0
                        \\
                        \iff & 
                        \fee \ammR[1] (\fee x_0 - x_0) - \fee \valSXa x_0 (\ammR[0] + \fee x_0) < 0
                        \\
                        \iff & 
                        \fee \ammR[1] x_0 (\fee - 1) - \fee \valSXa x_0 (\ammR[0] + \fee x_0) < 0
                    \end{align*}

                    Since we know that $\fee \valSXa x_0 (\ammR[0] + \fee x_0) > 0$ because all the terms are strictly positive, and $\fee \ammR[1] x_0 (\fee - 1) \leq 0 \iff \fee \leq 1$, then we finish the proof. 
                \end{proof}
                Now we can continue the proof:
                \begin{align*}
                    \X[\confGi]{\tokT[0],\tokT[1]} 
                    & = 
                    \lim_{z \rightarrow 0} \SX{z,\ammR[0] + x_0,\ammR[1] - \valSXa x_0} && \text{(by Def. of X)}
                    \\
                    & =
                    \frac{\fee(\ammR[1] - \alpha x_0)}{\ammR[0] + x_0}
                    \\
                    & < 
                    \frac{\fee \ammR[1]}{\ammR[0] + x_0} && \text{by Eq~\ref{eq:SX-less-SX-fee}}
                    \\
                    & = 
                    \SX{x_0, \ammR[0], \ammR[1]}
                \end{align*}
    \end{itemize}
\end{proofof}

\begin{proofof}{Lemma}{lem:fee:split-int-vs-ext-rate}
    Consider
    \begin{align*}
        y_1 & = \SX{x_1, \ammR[0], \ammR[1]} \cdot x_1 
        \\  & = \frac{\fee \ammR[1] x_1}{\ammR[0] + \fee x_1}
        \\
        y_2 & = \SX{x_2, \ammR[0] + x_1, \ammR[1] - y_1} \cdot x_2
        \\  & = \frac{\fee \ammR[0] \ammR[1] x_2}{(\ammR[0] + \fee x_1)(\ammR[0] + x_1 + \fee x_2)}
        \\
        y_3 & = \SX{x1 + x_2, \ammR[0] + x_1 + x_2, \ammR[1] - (y_1 + y_2)} \cdot (x_1 + x_2)
        \\  & = \SX{x_0, \ammR[0] + x_0, \ammR[1] - (y_1 + y_2)} \cdot x_0 
        \\  & = \frac{\fee \ammR[1] (x_1 + x_2)}{\ammR[0] + \fee x_1 + \fee x_2}
    \end{align*}

    Then, 
    \begin{align*}
        \X[{\confG[2]}]{\tokT[0],\tokT[1]} & = 
        \lim_{z \rightarrow 0} \SX{z,\ammR[0] + x_0,\ammR[1] - (y_1 + y_2)}
        && (\text{By Def.})
        \\
        & > \lim_{z \rightarrow 0} \SX{z,\ammR[0] + x_0,\ammR[1] - y_3}
        && (\text{By strict mono. proof below})
        \\
        & = \X[\confGi]{\tokT[0],\tokT[1]} = \X{\tokT[0],\tokT[1]}
        && (\text{By. Def. and Hyp.})
    \end{align*}

    We know that $z \leq z$ and $\ammR[0] + x_0 \leq \ammR[0] + x_0$, hence we only have to prove that $\ammR[1] - (y_1 + y_2) > \ammR[1] - y_3$:
    \begin{align*}
        & \ammR[1] - (y_1 + y_2) > \ammR[1] - y_3
        \\
        \iff & y_1 + y_2 - y_3 < 0
        \\
        \iff & \frac{\fee \ammR[1] x_1}{\ammR[0] + \fee x_1} + \frac{\fee \ammR[0] \ammR[1] x_2}{(\ammR[0] + \fee x_1)(\ammR[0] + x_1 + \fee x_2)}
        - \frac{\fee \ammR[1] (x_1 + x_2)}{\ammR[0] + \fee x_1 + \fee x_2} < 0
        \\
        \iff & \frac
            {\fee \ammR[1] x_1 (\ammR[0] + x_1 + \fee x_2)(\ammR[0] + \fee x_1 + \fee x_2) + \fee \ammR[0] \ammR[1] x_2 (\ammR[0] + \fee x_1 + \fee x_2)}
            {(\ammR[0] + \fee x_1)(\ammR[0] + x_1 + \fee x_2)(\ammR[0] + \fee x_1 + \fee x_2)}
            \\
            - & \frac{\fee \ammR[1] x_1 (\ammR[0] + \fee x_1)(\ammR[0] + x_1 + \fee x_2) + \fee \ammR[1] x_2 (\ammR[0] + \fee x_1)(\ammR[0] + x_1 + \fee x_2)}
            {(\ammR[0] + \fee x_1)(\ammR[0] + x_1 + \fee x_2)(\ammR[0] + \fee x_1 + \fee x_2)} < 0
        \\
        \iff & 
            \fee \ammR[1] x_1 (\ammR[0] + x_1 + \fee x_2)(\ammR[0] + \fee x_1 + \fee x_2) + \fee \ammR[0] \ammR[1] x_2 (\ammR[0] + \fee x_1 + \fee x_2) 
            \\
            - & \fee \ammR[1] x_1 (\ammR[0] + \fee x_1)(\ammR[0] + x_1 + \fee x_2) - \fee \ammR[1] x_2 (\ammR[0] + \fee x_1)(\ammR[0] + x_1 + \fee x_2) < 0
        \\
        \iff & 
            \fee \ammR[1] x_1 (\ammR[0] + x_1 + \fee x_2)((\ammR[0] + \fee x_1 + \fee x_2) - (\ammR[0] + \fee x_1)) + \fee \ammR[0] \ammR[1] x_2 (\ammR[0] + \fee x_2)
            \\
            + & \fee^2 \ammR[0] \ammR[1] x_1 x_2 - \fee \ammR[0] \ammR[1] x_1 x_2 - \fee \ammR[0] \ammR[1] x_2 (\ammR[0] + \fee x_2) - \fee^2 \ammR[1] x_1 x_2 (\ammR[0] + x_1 + \fee x_2) < 0
        \\
        \iff & 
            \fee^2 \ammR[1] x_1 x_2 (\ammR[0] + x_1 + \fee x_2) + \fee \ammR[0] \ammR[1] x_2 (\ammR[1] + \fee x_2) + \fee^2 \ammR[0] \ammR[1] x_1 x_2 - \fee \ammR[0] \ammR[1] x_1 x_2
            \\
            - & \fee \ammR[0] \ammR[1] x_2 (\ammR[0] + \fee x_2) - \fee^2 \ammR[1] x_1 x_2 (\ammR[0] + x_1 + \fee x_2) < 0
        \\
        \iff & 
            \fee^2 \ammR[0] \ammR[1] x_1 x_2 - \fee \ammR[0] \ammR[1] x_1 x_2 < 0
        \\
        \iff & 
            \fee \ammR[0] \ammR[1] x_1 x_2 (\fee - 1) < 0
        \\
        \iff &
            \fee - 1 < 0 
        \\
        \iff & \fee < 1
    \end{align*}
\end{proofof}

\begin{proofof}{Lemma}{lem:arbitrage:balance}
    Before starting the proof, we rewrite some terms: 
    \begin{align*}
        x_0 + \ammR[0] & = 
        \frac
        {-\sqrt{\exchO{\tokT[0]}} \ammR[0] (1 + \fee) + \sqrt{\ammR[0]} \sqrt{\exchO{\tokT[0]} \ammR[0] (-1 + \fee)^2 + 4 \exchO{\tokT[1]} \ammR[1] \fee^2}}
        {2 \sqrt{\exchO{\tokT[0]}} \fee} + \ammR[0]
        \\
        & = 
        \frac
        {-\sqrt{\exchO{\tokT[0]}} \ammR[0] (1 + \fee) + \sqrt{\ammR[0]} \sqrt{\exchO{\tokT[0]} \ammR[0] (-1 + \fee)^2 + 4 \exchO{\tokT[1]} \ammR[1] \fee^2} + 2 \sqrt{\exchO{\tokT[0]}} \fee \ammR[0]}
        {2 \sqrt{\exchO{\tokT[0]}} \fee}
        \\
        & = 
        \frac
        {-\sqrt{\exchO{\tokT[0]}} \ammR[0] (1 - \fee) + \sqrt{\ammR[0]} \sqrt{\exchO{\tokT[0]} \ammR[0] (-1 + \fee)^2 + 4 \exchO{\tokT[1]} \ammR[1] \fee^2}}
        {2 \sqrt{\exchO{\tokT[0]}} \fee}
        \\
        & = 
        \frac
        {\sqrt{\exchO{\tokT[0]}} \ammR[0] (\fee - 1) + \sqrt{\ammR[0]} \sqrt{\exchO{\tokT[0]} \ammR[0] (-1 + \fee)^2 + 4 \exchO{\tokT[1]} \ammR[1] \fee^2}} 
        {2 \sqrt{\exchO{\tokT[0]}} \fee}
        \numberthis \label{eq:x0-r0}
        \\
        \fee x_0 + \ammR[0] & = \fee \cdot
        \frac
        {-\sqrt{\exchO{\tokT[0]}} \ammR[0] (1 + \fee) + \sqrt{\ammR[0]} \sqrt{\exchO{\tokT[0]} \ammR[0] (-1 + \fee)^2 + 4 \exchO{\tokT[1]} \ammR[1] \fee^2}}
        {2 \sqrt{\exchO{\tokT[0]}} \fee} + \ammR[0]
        \\
        & = \frac
        {-\sqrt{\exchO{\tokT[0]}} \ammR[0] (1 + \fee) + \sqrt{\ammR[0]} \sqrt{\exchO{\tokT[0]} \ammR[0] (-1 + \fee)^2 + 4 \exchO{\tokT[1]} \ammR[1] \fee^2} + 2 \sqrt{\exchO{\tokT[0]}} \ammR[0]}
        {2 \sqrt{\exchO{\tokT[0]}}}
        \\
        & = 
        \frac
        {-\sqrt{\exchO{\tokT[0]}} \ammR[0] (\fee - 1) + \sqrt{\ammR[0]} \sqrt{\exchO{\tokT[0]} \ammR[0] (-1 + \fee)^2 + 4 \exchO{\tokT[1]} \ammR[1] \fee^2}}
        {2 \sqrt{\exchO{\tokT[0]}}} \numberthis \label{eq:fee-x0-r0}
    \end{align*}
    And let 
    \begin{align*}
        m & = \sqrt{\ammR[0]} \sqrt{\exchO{\tokT[0]} \ammR[0] (-1 + \fee)^2 + 4 \exchO{\tokT[1]} \ammR[1] \fee^2}
        \\
        \valSXa & = \SX{x_0, \ammR[0], \ammR[1]} = \frac{\fee \ammR[1]}{\ammR[0] + \fee x_0}
    \end{align*}
    Then, 
    \begin{align*}
        \X[\confGi]{\tokT[0], \tokT[1]} & = 
        \lim_{z \rightarrow 0} \SX{z,\ammR[0] + x_0,\ammR[1] -\valSXa x_0}  && (\text{By Def.})
        \\
        & = \frac
            {\fee (\ammR[1] - x_0 \cdot \frac
                {\fee \ammR[1]}
                {\ammR[0] + \fee x_0})}
            {\ammR[0] + x_0}
        \\
        & = \frac
            {\fee (\frac
                {\ammR[0]\ammR[1] + \fee \ammR[1]x_0 - \fee \ammR[1] x_0}
                {\ammR[0] + \fee x_0})}
            {\ammR[0] + x_0}
        \\
        & = \frac
            {\fee \ammR[0] \ammR[1]}
            {(\ammR[0] + x_0) (\ammR[0] + \fee x_0)}
        \\
        & = \frac
            {\fee \ammR[0] \ammR[1]}
            {
            (\frac
            {\sqrt{\exchO{\tokT[0]}} \ammR[0] (\fee - 1) + m}
            {2 \sqrt{\exchO{\tokT[0]}} \fee})
            (\frac
            {-\sqrt{\exchO{\tokT[0]}} \ammR[0] (\fee - 1) + m}
            {2 \sqrt{\exchO{\tokT[0]}}})}   && (\text{By Subst.})
        \\
        & = \frac
            {\fee \ammR[0] \ammR[1]}
            {\frac
                {(\sqrt{\exchO{\tokT[0]}} \ammR[0] (\fee - 1) + m)(-\sqrt{\exchO{\tokT[0]}} \ammR[0] (\fee - 1) + m)}
                {4 \exchO{\tokT[0]} \fee}}
        \\
        & = \frac
            {\fee \ammR[0] \ammR[1] (4 \exchO{\tokT[0]} \fee)}
            {m^2 - (\sqrt{\exchO{\tokT[0]}} \ammR[0] (\fee - 1))^2}
        \\
        & = \frac
            {\fee \ammR[0] \ammR[1] (4 \exchO{\tokT[0]} \fee)}
            {\ammR[0] (\exchO{\tokT[0]} \ammR[0] (\fee - 1)^2 + 4 \exchO{\tokT[1]} \ammR[1] \fee ^2) - \exchO{\tokT[0]} \ammR[0]^2 (\fee - 1)^2}
        \\
        & = \frac
            {\fee \ammR[0] \ammR[1] (4 \exchO{\tokT[0]} \fee)}
            {\exchO{\tokT[0]} \ammR[0]^2 (\fee - 1)^2 + 4 \exchO{\tokT[1]} \ammR[0] \ammR[1] \fee ^2 - \exchO{\tokT[0]} \ammR[0]^2 (\fee - 1)^2}
        \\
        & = \frac
            {\fee \ammR[0] \ammR[1] (4 \exchO{\tokT[0]} \fee)}
            { 4 \exchO{\tokT[1]} \ammR[0] \ammR[1] \fee ^2 }
        \\ & = 
        \frac{\exchO{\tokT[0]}}{\exchO{\tokT[1]}}
    \end{align*}
\end{proofof}

\begin{proofof}{Lemma}{lem:arbitrage:balance-unique}
    
\end{proofof}

\begin{proofof}{Lemma}{lem:fee:equil-vs-gain}
    \begin{itemize}
        \item Case $x < x_0$: 

            Let $x_1 > 0$ be such that $x_0 = x + x_1$. Since $\SX{}$ is output bounded, then $\txT(x_0)$, $\txT(x)$ and $\txT(x_1)$ are all enabled, in particular : 

            \begin{align*}
                & \confG \xrightarrow{\quad\quad\txT(x_0)\quad\quad} \confGi
                \\
                & \confG \xrightarrow{\txT(x)} \confG[1]\xrightarrow{\txT(x_1)} \confG[2]
            \end{align*}

            By Lemma ~\ref{lem:fee:swap-gain:additive} we know that: 

            \[
            \gain[\confG]{\pmvA}{\txT(x_0)} 
            \; = \;
            \gain[\confG]{\pmvA}{\txT(x)} + \gain[{\confG[1]}]{\pmvA}{\txT(x_1)} + \epsilon_{\fee}
            \]

            Since we know that $\epsilon_{\fee} > 0$ we just have to prove that $\gain[{\confG[1]}]{\pmvA}{\txT(x_1)} > 0$. To do that, we first prove that $\SX{x_1, \ammR[0] + x, \ammR[1] - \valSXa x} > \X{\tokT[0], \tokT[1]}$ where $\valSXa = \SX{x, \ammR[0], \ammR[1]}$: 
            \begin{align*}
                    \X{\tokT[0], \tokT[1]}
                & < \X[{\confG[2]}]{\tokT[0], \tokT[1]} && \text{by ~\ref{lem:fee:split-int-vs-ext-rate}}
                \\
                & < \SX{x_1, \ammR[0] + x, \ammR[1] - \valSXa x}    && \text{by ~\ref{lem:fee:sx-rate-vs-int-rate}}
            \end{align*}

            So, by applying Lemma ~\ref{lem:fee:swap:gain-SX-X-fee} we can finish the proof.

        \item Case $x > x_0$: 
             Let $x_1 > 0$ be such that $x = x_0 + x_1$. Since $\SX{}$ is output bounded, then $\txT(x_0)$, $\txT(x)$ and $\txT(x_1)$ are all enabled, in particular : 

            \begin{align*}
                & \confG \xrightarrow{\quad\quad\txT(x)\quad\quad} \confG[1]
                \\
                & \confG \xrightarrow{\txT(x_0)} \confGi\xrightarrow{\txT(x_1)} \confG[2]
            \end{align*}

            By Lemma ~\ref{lem:fee:swap-gain:additive} we know that: 

            \[
            \gain[\confG]{\pmvA}{\txT(x)} 
            \; = \;
            \gain[\confG]{\pmvA}{\txT(x_0)} + \gain[{\confGi}]{\pmvA}{\txT(x_1)} + \epsilon_{\fee}
            \]

            To finish our proof, we need to prove that $\gain[{\confGi}]{\pmvA}{\txT(x_1)} + \epsilon_{\fee} < 0$. Since we know that $\epsilon_{\fee} > 0$, we have first to see if $\gain[{\confGi}]{\pmvA}{\txT(x_1)} < 0$, otherwise it is always false. So, we want to show that $\SX{x_1, \ammR[0] + x, \ammR[1] - \valSXa x} < \X{\tokT[0], \tokT[1]}$ where $\valSXa = \SX{x_0, \ammR[0], \ammR[1]}$: 
                \begin{align*}
                    \X{\tokT[0], \tokT[1]}
                & = \X[{\confGi}]{\tokT[0], \tokT[1]}
                \\
                & < \SX{x_1, \ammR[0] + x_0, \ammR[1] - \valSXa x_0}    && \text{by ~\ref{lem:fee:sx-rate-vs-int-rate}}
            \end{align*}

            So, by applying Lemma ~\ref{lem:fee:swap:gain-SX-X-fee} we know that $\gain[{\confGi}]{\pmvA}{\txT(x_1)} < 0$, hence we can continue the proof.

            \begin{proof}
            
                Before proceeding with the proof, we point out the values of some terms that we will be using during the proof: 

                \begin{align*}
                    x_1 \valSXb
                    & = x_1 \cdot \frac
                            {\fee \Big ( \ammR[1] - \frac
                                {\fee \ammR[1]}
                                {\ammR[0] + \fee x_0} \cdot x_0
                            \Big )}
                            {\ammR[0] + x_0 + \fee x_1 }
                      = x_1 \cdot \frac
                            {\fee \Big (\frac
                                {\ammR[1] (\ammR[0] + \fee x_0) - \fee \ammR[1] x_0}
                                {\ammR[0] + \fee x_0}
                            \Big )}
                            {\ammR[0] + x_0 + \fee x_1 }
                    \\
                    & = x_1 \cdot \frac
                            {\fee \Big (\frac
                                {\ammR[1] \ammR[0] + \fee \ammR[1] x_0 - \fee \ammR[1] x_0}
                                {\ammR[0] + \fee x_0}
                            \Big )}
                            {\ammR[0] + x_0 + \fee x_1 }
                     = \frac
                            {\fee \ammR[0] \ammR[1] x_1}
                            {(\ammR[0] + x_0 + \fee x_1) (\ammR[0] + \fee x_0)}
                \\
                \\
                    (\valSXa x_0 + \valSXb x_1)
                    & = 
                    \frac{\fee \ammR[1] x_0}{\ammR[0] + \fee x_0} 
                    + 
                    \frac
                        {\fee \ammR[0] \ammR[1] x_1}
                        {(\ammR[0] + x_0 + \fee x_1) (\ammR[0] + \fee x_0)}
                    = 
                    \frac
                        {\fee \ammR[1] x_0 (\ammR[0] + x_0 + \fee x_1) + \fee \ammR[0] \ammR[1] x_1}
                        {(\ammR[0] + x_0 + \fee x_1) (\ammR[0] + \fee x_0)}
                    \\
                    & = 
                    \frac
                        {\fee \ammR[0] \ammR[1] x_0 + \fee \ammR[1] x_0^2 + \fee^2 \ammR[1] x_0 x_1 + \fee \ammR[0] \ammR[1] x_1} 
                        {(\ammR[0] + x_0 + \fee x_1) (\ammR[0] + \fee x_0)}
                    = 
                    \frac
                        {\fee \ammR[1] (\ammR[0] x_0 + x_0^2 + \fee x_0 x_1 + \ammR[0] x_1)} 
                        {(\ammR[0] + x_0 + \fee x_1) (\ammR[0] + \fee x_0)}
                \\
                \\
                    (z-1)
                    & = \frac
                        {(x_0 + x_1)(\ammR[0] + \fee x_0)(\ammR[0] + x_0 + \fee x_1)}
                        {(\ammR[0] + \fee x_0 + \fee x_1)(x_0^2 + \ammR[0] x_0 +\fee x_0 x_1 + \ammR[0] x_1)} - 1
                    \\
                    & = \frac
                        {(\ammR[0] x_0 + \fee x_0 ^ 2 + \ammR[0] x_1 + \fee x_0 x_1)(\ammR[0] + x_0 + \fee x_1)}
                        {(\ammR[0] + \fee x_0 + \fee x_1)(x_0^2 + \ammR[0] x_0 +\fee x_0 x_1 + \ammR[0] x_1)}
                    \\
                    & -
                        \frac
                        {(\ammR[0] + \fee x_0 + \fee x_1 )(\ammR[0] x_0 + x_0 ^ 2 + \ammR[0] x_1 + \fee x_0 x_1)}
                        {(\ammR[0] + \fee x_0 + \fee x_1)(x_0^2 + \ammR[0] x_0 +\fee x_0 x_1 + \ammR[0] x_1)}
                    \\
                    & = - \frac
                        {\ammR[0] x_0 x_1 (\fee - 1)}
                        {(\ammR[0] + \fee x_0 + \fee x_1)(x_0^2 + \ammR[0] x_0 +\fee x_0 x_1 + \ammR[0] x_1)}
                \\
                \\
                    x_1 \valSXb & + (z - 1)(\valSXa x_0 + \valSXb x_1)
                    \\
                    & = \frac
                            {\fee \ammR[0] \ammR[1] x_1}
                            {(\ammR[0] + x_0 + \fee x_1) (\ammR[0] + \fee x_0)} 
                            - \frac
                            {\ammR[0] \ammR[1] x_0 x_1 (\fee - 1) \fee}
                            {(\ammR[0] + \fee x_0 + \fee x_1) (\ammR[0] + \fee x_0) ( \ammR[0] + x_0 + \fee x_1)}
                    \\
                    & = 
                        \frac
                            {\fee \ammR[0] \ammR[1] x_1 (\ammR[0] + \fee x_0 + \fee x_1) - \ammR[0] \ammR[1] x_0 x_1 (\fee - 1) \fee}
                            {(\ammR[0] + \fee x_0 + \fee x_1) (\ammR[0] + \fee x_0) ( \ammR[0] + x_0 + \fee x_1)}
                    \\
                    & = 
                        \frac
                            {\fee \ammR[0] \ammR[1] x_1 ((\ammR[0] + \fee x_0 + \fee x_1) - x_0 (\fee - 1))}
                            {(\ammR[0] + \fee x_0 + \fee x_1) (\ammR[0] + \fee x_0) ( \ammR[0] + x_0 + \fee x_1)}
                    \\
                    & = 
                        \frac
                            {\fee \ammR[0] \ammR[1] x_1 (\ammR[0] + x_0 + \fee x_1)}
                            {(\ammR[0] + \fee x_0 + \fee x_1) (\ammR[0] + \fee x_0) ( \ammR[0] + x_0 + \fee x_1)}
                    \\
                    & = 
                        \frac
                            {\fee \ammR[0] \ammR[1] x_1}
                            {(\ammR[0] + \fee x_0 + \fee x_1) (\ammR[0] + \fee x_0)}
                    \numberthis 
                    \label{eq:x1z}
                \\
                \\
                    \X[{\confGi}]{\tokT[0], \tokT[1]}
                    & = \lim_{z \rightarrow 0} \SX{z,\ammR[0] + x_0,\ammR[1] -\valSXa x_0}
                    \\
                    & = \frac
                        {\fee (\ammR[1] - \valSXa x_0)}
                        {\ammR[0] + x_0}
                      = \frac
                        {\fee \Big (\frac
                                {\ammR[1] (\ammR[0] + \fee x_0) - \fee \ammR[1] x_0}
                                {\ammR[0] + \fee x_0}
                            \Big )}
                        {\ammR[0] + x_0}
                      = \frac
                            {\fee \ammR[0] \ammR[1]}
                            {(\ammR[0] + \fee x_0) (\ammR[0] + x_0)}
                \end{align*}

                With these values in mind, we can continue the proof: 
                \begin{align*}
                    & \gain[{\confGi}]{\pmvA}{\txT(x_1)} + \epsilon_{\fee} < 0
                    \\
                    & \iff
                    x_1 (\valSXb \exchO{\tokT[1]} - \exchO{\tokT[0]}) + \exchO{\tokT[1]}(z - 1) (\valSXa x_0 + \valSXb x_1) < 0
                    \\
                    & \iff 
                    x_1 \valSXb \exchO{\tokT[1]} - x_1 \exchO{\tokT[0]} + \exchO{\tokT[1]}(z - 1) (\valSXa x_0 + \valSXb x_1) < 0
                    \\
                    & \iff 
                    \exchO{\tokT[1]} (x_1 \valSXb + (z - 1) (\valSXa x_0 + \valSXb x_1)) < x_1 \exchO{\tokT[0]}
                    \\ 
                    & \iff
                    \frac
                        {x_1 \valSXb + (z - 1) (\valSXa x_0 + \valSXb x_1)}
                        {x_1}
                    < \frac{\exchO{\tokT[0]}}{\exchO{\tokT[1]}}
                    && \text{by } x_1 > 0 \land \exchO{\tokT[1]} > 0
                    \\ 
                    & \iff
                    \frac
                        {x_1 \valSXb + (z - 1) (\valSXa x_0 + \valSXb x_1)}
                        {x_1}
                    < \X[\confGi]{\tokT[0], \tokT[1]}
                    \\
                    & \iff 
                    \frac
                        {\frac
                            {\fee \ammR[0] \ammR[1] x_1}
                            {(\ammR[0] + \fee x_0 + \fee x_1) (\ammR[0] + \fee x_0)}}
                        {x_1}
                    < \X[\confGi]{\tokT[0], \tokT[1]}
                    \\
                    & \iff 
                    \frac
                        {\fee \ammR[0] \ammR[1]}
                        {(\ammR[0] + \fee x_0 + \fee x_1) (\ammR[0] + \fee x_0)}
                    < \frac
                            {\fee \ammR[0] \ammR[1]}
                            {(\ammR[0] + \fee x_0) (\ammR[0] + x_0)}
                    \\
                    & \iff
                    \frac
                        {\fee \ammR[0] \ammR[1] (\ammR[0] + x_0) - \fee \ammR[0] \ammR[1] (\ammR[0] + \fee x_0 + \fee x_1)}
                        {(\ammR[0] + \fee x_0 + \fee x_1) (\ammR[0] + \fee x_0) (\ammR[0] + x_0)} < 0
                    \\
                    & \iff
                        \fee \ammR[0] \ammR[1] (\ammR[0] + x_0) - \fee \ammR[0] \ammR[1] (\ammR[0] + \fee x_0 + \fee x_1) < 0  && \text{by Den. } > 0
                    \\
                    & \iff
                        \fee \ammR[0] \ammR[1] (\ammR[0] + x_0 - \ammR[0] - \fee x_0 - \fee x_1) < 0
                    \\
                    & \iff
                        x_0 - \fee x_0 - \fee x_1 < 0    && \text{by } \fee \ammR[0] \ammR[1] < 0
                    \\
                    & \iff
                        \fee (x_0 + x_1) - x_0 > 0
                    \\
                    & \iff
                        \fee > \frac{x_0}{x_0 + x_1}
                    \\
                    & \iff
                    x_1 > \frac{x_0(1 - \fee)}{\fee}
                \end{align*}

                So, this is only true when: 
                \begin{align*}
                    & \iff
                    x_1 > \frac{x_0(1 - \fee)}{\fee}
                    \\
                    & \iff
                    x - x_0 > \frac{x_0(1 - \fee)}{\fee}
                    \\
                    & \iff
                    x > \frac{x_0(1 - \fee) + \fee x_0}{\fee}
                    \\
                    & \iff
                    x > \frac{x_0}{\fee}
                \end{align*}
            \end{proof}
    \end{itemize}
\end{proofof}

\begin{proofof}{Lemma}{lem:fee:max-gain} 

    Let $x_1 > 0$ be such that $x_{max} = x_0 + x_1$. Then, as we stated in the proof of Theorem ~\ref{lem:fee:equil-vs-gain}, by Lemma ~\ref{lem:fee:swap-gain:additive} we know that: 

        \[
        \gain[\confG]{\pmvA}{\txT(x_{max})} 
        \; = \;
        \gain[\confG]{\pmvA}{\txT(x_0)} + \gain[{\confGi}]{\pmvA}{\txT(x_1)} + \epsilon_{\fee}
        \]
    We have proven in Theorem ~\ref{lem:fee:equil-vs-gain} that for $x < \frac{x_0}{\fee}$, $\gain[{\confGi}]{\pmvA}{\txT(x_1)} + \epsilon_{\fee} > 0$. Since $x_0$ is fixed, we want to find the maximum of the function: 
    \begin{equation}
        f(x_1) = \gain[\confG]{\pmvA}{\txT(x_0)} + \gain[{\confGi}]{\pmvA}{\txT(x_1)} + \epsilon_{\fee}
    \end{equation}

    To find the maximum, first we look at the function values at the extremes of its domain: 

    \begin{align*}
        \lim_{x_1 \rightarrow 0} f(x_1) & = 
        \lim_{x_1 \rightarrow 0} \gain[\confG]{\pmvA}{\txT(x_0)} + \gain[{\confGi}]{\pmvA}{\txT(x_1)} + \epsilon_{\fee}
        \\
        & = \lim_{x_1 \rightarrow 0} \gain[\confG]{\pmvA}{\txT(x_0)} +
        x_1 \valSXb \exchO{\tokT[1]} - x_1 \exchO{\tokT[0]} + \exchO{\tokT[1]} (z - 1)(\valSXa x_0 + \valSXb x_1)
        \\
        & = \lim_{x_1 \rightarrow 0} \gain[\confG]{\pmvA}{\txT(x_0)}
        - x_1 \exchO{\tokT[0]} + \exchO{\tokT[1]} ((z-1)(\valSXa x_0 + \valSXb x_1) + \valSXb x_1)
        \\
        & = \lim_{x_1 \rightarrow 0} \gain[\confG]{\pmvA}{\txT(x_0)}
            - x_1 \exchO{\tokT[0]} + 
            \frac
                {\exchO{\tokT[1]} \fee \ammR[0] \ammR[1] x_1}
                {(\ammR[0] + \fee x_0) (\ammR[0] + \fee x_0 + \fee x_1)}
        && (\text{By~\ref{eq:x1z}})
        \\
        & = \gain[\confG]{\pmvA}{\txT(x_0)} - 0 + \frac{0}{(\ammR[0] + \fee x_0) (\ammR[0] + \fee x_0)}
        \\
        & = \gain[\confG]{\pmvA}{\txT(x_0)}
        \\
        \phantom{space} \\
        \lim_{x_1 \rightarrow \infty} f(x_1) & = 
        \lim_{x_1 \rightarrow \infty} \gain[\confG]{\pmvA}{\txT(x_0)} + \gain[{\confGi}]{\pmvA}{\txT(x_1)} + \epsilon_{\fee}
        \\
        & = \lim_{x_1 \rightarrow \infty} \gain[\confG]{\pmvA}{\txT(x_0)}
            - x_1 \exchO{\tokT[0]} + 
            \frac
                {\exchO{\tokT[1]} \fee \ammR[0] \ammR[1] x_1}
                {(\ammR[0] + \fee x_0) (\ammR[0] + \fee x_0 + \fee x_1)}
        \\
        & = \lim_{x_1 \rightarrow \infty} \gain[\confG]{\pmvA}{\txT(x_0)}
        - x_1 \exchO{\tokT[0]} + 
        \frac
                {x_1 \exchO{\tokT[1]} \fee \ammR[0] \ammR[1]}
                {x_1 (\ammR[0] + \fee x_0) (\frac{\ammR[0]}{x_1} + \frac{\fee x_0}{x_1} + \fee)}
        \\
        & = \gain[\confG]{\pmvA}{\txT(x_0)}
            - \infty \exchO{\tokT[0]} + 
            \frac
                {\exchO{\tokT[1]} \fee \ammR[0] \ammR[1]}
                {(\ammR[0] + \fee x_0) (\frac{\ammR[0]}{x_1} + \frac{\fee x_0}{x_1} + \fee)}
        \\ 
        & = - \infty
    \end{align*}

    Now, we have to study the critical points of the function. To do that, we first have to compute the first derivative of $f(x_1)$:
    \begin{align*}
        f'(x_1) = &  \Big [
             - x_1 \exchO{\tokT[0]} + 
            \frac
                {\exchO{\tokT[1]} \fee \ammR[0] \ammR[1] x_1}
                {(\ammR[0] + \fee x_0) (\ammR[0] + \fee x_0 + \fee x_1)}
                \Big ]'
        \\
        = &  -\exchO{\tokT[0]} + \Big [ 
            \frac
                {\exchO{\tokT[1]} \fee \ammR[0] \ammR[1] x_1}
                {(\ammR[0] + \fee x_0) (\ammR[0] + \fee x_0 + \fee x_1)}
                \Big ]'
        \\
        = &  -\exchO{\tokT[0]} +
            \frac
                {[\exchO{\tokT[1]} \fee \ammR[0] \ammR[1] x_1]' (\ammR[0] + \fee x_0) (\ammR[0] + \fee x_0 + \fee x_1)}
                {((\ammR[0] + \fee x_0) (\ammR[0] + \fee x_0 + \fee x_1))^2}
                \\
            & -  \frac
                {(\exchO{\tokT[1]} \fee \ammR[0] \ammR[1] x_1)[(\ammR[0] + \fee x_0) (\ammR[0] + \fee x_0 + \fee x_1)]'}
                {((\ammR[0] + \fee x_0) (\ammR[0] + \fee x_0 + \fee x_1))^2}
        \\
        = &  -\exchO{\tokT[0]} +
            \frac
                {\exchO{\tokT[1]} \fee \ammR[0] \ammR[1] (\ammR[0] + \fee x_0) (\ammR[0] + \fee x_0 + \fee x_1)}
                {((\ammR[0] + \fee x_0) (\ammR[0] + \fee x_0 + \fee x_1))^2}
                \\
            & -  \frac
                {(\exchO{\tokT[1]} \fee \ammR[0] \ammR[1] x_1) \fee (\ammR[0] + \fee x_0)}
                {((\ammR[0] + \fee x_0) (\ammR[0] + \fee x_0 + \fee x_1))^2}
        \\
        = &  -\exchO{\tokT[0]} +
            \frac
                {\exchO{\tokT[1]} \fee \ammR[0] \ammR[1] \Big ( (\ammR[0] + \fee x_0) (\ammR[0] + \fee x_0 + \fee x_1) - \fee x_1 (\ammR[0] + \fee x_0) \Big )}
                {(\ammR[0] + \fee x_0)^2 (\ammR[0] + \fee x_0 + \fee x_1)^2}
        \\
        = &  -\exchO{\tokT[0]} +
            \frac
                {\exchO{\tokT[1]} \fee \ammR[0] \ammR[1] (\ammR[0] + \fee x_0) \Big ( \ammR[0] + \fee x_0 + \fee x_1 - \fee x_1 \Big )}
                {(\ammR[0] + \fee x_0)^2 (\ammR[0] + \fee x_0 + \fee x_1)^2}
        \\
        = &  -\exchO{\tokT[0]} +
            \frac
                {\exchO{\tokT[1]} \fee \ammR[0] \ammR[1] (\ammR[0] + \fee x_0) ( \ammR[0] + \fee x_0)}
                {(\ammR[0] + \fee x_0)^2 (\ammR[0] + \fee x_0 + \fee x_1)^2}
        \\
        = &  -\exchO{\tokT[0]} +
            \frac
                {\exchO{\tokT[1]} \fee \ammR[0] \ammR[1]}
                {(\ammR[0] + \fee x_0 + \fee x_1)^2}
    \end{align*}

    Now, we want to prove that 
    \begin{equation}
        x' =
        \frac
            {-\sqrt{\exchO{\tokT[0]}} \ammR[0] - \sqrt{\exchO{\tokT[0]}} \fee x_0 + \sqrt{\exchO{\tokT[1]} \fee \ammR[0] \ammR[1]}}
            {\sqrt{\exchO{\tokT[0]}} \fee}
    \end{equation}
    is a critical point of $f(x_1)$, i.e. $f'(x') = 0$. 
    First, let's rewrite: 
    \begin{align*}
        \fee x' = & 
        \fee \cdot \frac
            {-\sqrt{\exchO{\tokT[0]}} \ammR[0] - \sqrt{\exchO{\tokT[0]}} \fee x_0 + \sqrt{\exchO{\tokT[1]} \fee \ammR[0] \ammR[1]}}
            {\sqrt{\exchO{\tokT[0]}} \fee}
        \\
        = & 
        \frac
            {-\sqrt{\exchO{\tokT[0]}} \ammR[0] - \sqrt{\exchO{\tokT[0]}} \fee x_0 + \sqrt{\exchO{\tokT[1]} \fee \ammR[0] \ammR[1]}}
            {\sqrt{\exchO{\tokT[0]}}}
        \\
        = & 
        \frac
            {-\sqrt{\exchO{\tokT[0]}} (\ammR[0] + \fee x_0)}
            {\sqrt{\exchO{\tokT[0]}}} + 
        \frac
            {\sqrt{\exchO{\tokT[1]} \fee \ammR[0] \ammR[1]}}
            {\sqrt{\exchO{\tokT[0]}}}
        \\
        = & 
        - (\ammR[0] + \fee x_0)
            + 
        \frac
            {\sqrt{\exchO{\tokT[1]} \fee \ammR[0] \ammR[1]}}
            {\sqrt{\exchO{\tokT[0]}}} \numberthis \label{eq:feex'}
    \end{align*}

    Finally, 

    \begin{align*}
        f'(x') = & 
        -\exchO{\tokT[0]} +
            \frac
                {\exchO{\tokT[1]} \fee \ammR[0] \ammR[1]}
                {(\ammR[0] + \fee x_0 + \fee x')^2}
        \\
        = & 
        -\exchO{\tokT[0]} +
            \frac
                {\exchO{\tokT[1]} \fee \ammR[0] \ammR[1]}
                {(\ammR[0] + \fee x_0 - (\ammR[0] + \fee x_0)
                    + 
                    \frac
                    {\sqrt{\exchO{\tokT[1]} \fee \ammR[0] \ammR[1]}}
                    {\sqrt{\exchO{\tokT[0]}}})^2}
        \\
        = & 
        -\exchO{\tokT[0]} +
            \frac
                {\exchO{\tokT[1]} \fee \ammR[0] \ammR[1]}
                {(\frac
                    {\sqrt{\exchO{\tokT[1]} \fee \ammR[0] \ammR[1]}}
                    {\sqrt{\exchO{\tokT[0]}}})^2}
        \\
        = & 
        -\exchO{\tokT[0]} +
            \frac
                {\exchO{\tokT[1]} \fee \ammR[0] \ammR[1]}
                {\frac
                    {\exchO{\tokT[1]} \fee \ammR[0] \ammR[1]}
                    {\exchO{\tokT[0]}}}
        \\
        = & 
        -\exchO{\tokT[0]} + \exchO{\tokT[0]} = 0
    \end{align*}

    Now, since we know that this is either a local maximum or a local minimum, we only have to study the sign of the second derivative of $f(x_1)$ to determine which is the case. Let's compute it first: 

    \begin{align*}
        f''(x_1) & = \Big [  -\exchO{\tokT[0]} +
                \frac
                {\exchO{\tokT[1]} \fee \ammR[0] \ammR[1]}
                {(\ammR[0] + \fee x_0 + \fee x_1)^2}
            \Big ]'
        \\
        & = \Big [\frac
                {\exchO{\tokT[1]} \fee \ammR[0] \ammR[1]}
                {(\ammR[0] + \fee x_0 + \fee x_1)^2}  \Big ]'
        \\ 
        & = \exchO{\tokT[1]} \fee \ammR[0] \ammR[1] \cdot
            \Big [(\ammR[0] + \fee x_0 + \fee x_1)^{-2}  \Big ]'
        \\ 
        & = \exchO{\tokT[1]} \fee \ammR[0] \ammR[1] \cdot (-2) * (\ammR[0] + \fee x_0 + \fee x_1)^{-3} * [(\ammR[0] + \fee x_0 + \fee x_1)]'
        \\ 
        & = \exchO{\tokT[1]} \fee \ammR[0] \ammR[1] \cdot (-2) * (\ammR[0] + \fee x_0 + \fee x_1)^{-3} * \fee
        \\ 
        & = -2 \exchO{\tokT[1]} \fee^2 \ammR[0] \ammR[1] * (\ammR[0] + \fee x_0 + \fee x_1)^{-3}
    \end{align*}
    Since $x_1 > 0$ and all the other constant are also strictly positive, then it is easy to see that this second derivative is always negative for $x_1 > 0$, hence in the function's domain, the function is concave. This is enough to prove that $x'$ is a global maximum over the functions domain \cite{wiki:concave_function}

    Unfortunately, due to the limitation of the Lean 4 implementation of the model, it is not possible to conduct the proof this way in the theorem prover. Therefore, we present also the proof outlined in the implementation.

    Again, consider the function
    \begin{align*}
        f(x_1) & = \gain[\confG]{\pmvA}{\txT(x_0)} + \gain[{\confGi}]{\pmvA}{\txT(x_1)} + \epsilon_{\fee}
        \\
        & = \gain[\confG]{\pmvA}{\txT(x_0)} + x_1 \valSXb \exchO{\tokT[1]} - x_1 \exchO{\tokT[0]} + \exchO{\tokT[1]} (z - 1)(\valSXa x_0 + \valSXb x_1)
        \\ 
        & = \gain[\confG]{\pmvA}{\txT(x_0)} - x_1 \exchO{\tokT[0]} + \exchO{\tokT[1]} ((z-1)(\valSXa x_0 + \valSXb x_1) + \valSXb x_1)
        \\
        & = \gain[\confG]{\pmvA}{\txT(x_0)}
         - x_1 \exchO{\tokT[0]} + 
            \frac
                {\exchO{\tokT[1]} \fee \ammR[0] \ammR[1] x_1}
                {(\ammR[0] + \fee x_0) (\ammR[0] + \fee x_0 + \fee x_1)}
        && (\text{By~\ref{eq:x1z}})
    \end{align*}

    We want to prove that 

    \begin{equation}
        \label{eq:partial-xmax}
        x' =
        \frac
            {-\sqrt{\exchO{\tokT[0]}} \ammR[0] - \sqrt{\exchO{\tokT[0]}} \fee x_0 + \sqrt{\exchO{\tokT[1]} \fee \ammR[0] \ammR[1]}}
            {\sqrt{\exchO{\tokT[0]}} \fee}
    \end{equation}

    Is the global maximum of $f(x_1)$, \ie 

    \[
        \forall x \neq x', f(x') > f(x)
    \]

    First, let's rewrite the inequality: 
    \begin{align*}
        & 
        f(x') > f(x) 
        \\ \iff  &
        \gain[\confG]{\pmvA}{\txT(x_0)}
         - x' \exchO{\tokT[0]} + 
            \frac
                {\exchO{\tokT[1]} \fee \ammR[0] \ammR[1] x'}
                {(\ammR[0] + \fee x_0) (\ammR[0] + \fee x_0 + \fee x')} >
        \\ & 
        \gain[\confG]{\pmvA}{\txT(x_0)}
         - x \exchO{\tokT[0]} + 
            \frac
                {\exchO{\tokT[1]} \fee \ammR[0] \ammR[1] x}
                {(\ammR[0] + \fee x_0) (\ammR[0] + \fee x_0 + \fee x)}
        \\ \iff &
         - x' \exchO{\tokT[0]} + 
            \frac
                {\exchO{\tokT[1]} \fee \ammR[0] \ammR[1] x'}
                {(\ammR[0] + \fee x_0) (\ammR[0] + \fee x_0 + \fee x')} >
        \\ &
         - x \exchO{\tokT[0]} + 
            \frac
                {\exchO{\tokT[1]} \fee \ammR[0] \ammR[1] x}
                {(\ammR[0] + \fee x_0) (\ammR[0] + \fee x_0 + \fee x)}
        \\ \iff & 
            \frac
                {\exchO{\tokT[1]} \fee \ammR[0] \ammR[1] x'}
                {(\ammR[0] + \fee x_0) (\ammR[0] + \fee x_0 + \fee x')}
            - x' \exchO{\tokT[0]}
            - \Big ( 
            \frac
                {\exchO{\tokT[1]} \fee \ammR[0] \ammR[1] x}
                {(\ammR[0] + \fee x_0) (\ammR[0] + \fee x_0 + \fee x)}
            - x \exchO{\tokT[0]}
            \Big ) > 0
    \end{align*}

    Now let $A = \ammR[0] + \fee x_0$. Then, the inequality becomes

    \begin{align*}
            & 
            \frac
                {\exchO{\tokT[1]} \fee \ammR[0] \ammR[1] x'}
                {(\ammR[0] + \fee x_0) (\ammR[0] + \fee x_0 + \fee x')}
            - x' \exchO{\tokT[0]}
            - \Big ( 
            \frac
                {\exchO{\tokT[1]} \fee \ammR[0] \ammR[1] x}
                {(\ammR[0] + \fee x_0) (\ammR[0] + \fee x_0 + \fee x)}
            - x \exchO{\tokT[0]}
            \Big ) > 0
            \\ \iff & 
            \frac
                {\exchO{\tokT[1]} \fee \ammR[0] \ammR[1] x'}
                {A (A + \fee x')}
            - x' \exchO{\tokT[0]}
            - \Big ( 
            \frac
                {\exchO{\tokT[1]} \fee \ammR[0] \ammR[1] x}
                {A (A + \fee x)}
            - x \exchO{\tokT[0]}
            \Big ) > 0
            \\ \iff & 
            \frac
                {\exchO{\tokT[1]} \fee \ammR[0] \ammR[1] x' 
                - x' \exchO{\tokT[0]} A (A + \fee x')}
                {A (A + \fee x')}
            - \Big ( 
            \frac
                {\exchO{\tokT[1]} \fee \ammR[0] \ammR[1] x
                - x \exchO{\tokT[0]} A (A + \fee x)}
                {A (A + \fee x)}
            \Big ) > 0
    \end{align*}

    We define the terms 
    $D(x) = A (A + \fee x)$ and $N(x) = \exchO{\tokT[1]} \fee \ammR[0] \ammR[1] x - \exchO{\tokT[0]} x A (A + \fee x)$ to simplify the expression above: 
    \begin{align*}
        & 
            \frac
                {\exchO{\tokT[1]} \fee \ammR[0] \ammR[1] x' 
                - x' \exchO{\tokT[0]} A (A + \fee x')}
                {A (A + \fee x')}
            - \Big ( 
            \frac
                {\exchO{\tokT[1]} \fee \ammR[0] \ammR[1] x
                - x \exchO{\tokT[0]} A (A + \fee x)}
                {A (A + \fee x)}
            \Big ) > 0
        \\ \iff & 
        \frac{N(x')}{D(x')} - \frac{N(x)}{D(x)} > 0
        \\ \iff & 
        N(x')D(x) - N(x)D(x') > 0
        \\ \iff & 
        \Big (\exchO{\tokT[1]} \fee \ammR[0] \ammR[1] x' 
                - x' \exchO{\tokT[0]} A (A + \fee x') \Big )(A (A + \fee x)) 
        \\  & - 
        \Big ( \exchO{\tokT[1]} \fee \ammR[0] \ammR[1] x 
                - x \exchO{\tokT[0]} A (A + \fee x) \Big )(A (A + \fee x')) > 0
        \\ \iff &
        A \Big [ \Big (\exchO{\tokT[1]} \fee \ammR[0] \ammR[1] x' 
                - x' \exchO{\tokT[0]} A (A + \fee x') \Big )(A + \fee x)
        \\  & -
        \Big ( \exchO{\tokT[1]} \fee \ammR[0] \ammR[1] x 
                - x \exchO{\tokT[0]} A (A + \fee x) \Big )(A + \fee x') \Big ] > 0 \numberthis \label{eq:D}
    \end{align*}

    Let this term be 
    \[
    \Delta(x) = A \Delta^*(x)
    \]

    Now, observe that 
    \begin{align*}
        \Delta^*(x) & = 
        \Big (\exchO{\tokT[1]} \fee \ammR[0] \ammR[1] x' 
                - x' \exchO{\tokT[0]} A (A + \fee x') \Big )(A + \fee x)
        \\  & - 
        \Big ( \exchO{\tokT[1]} \fee \ammR[0] \ammR[1] x 
                - x \exchO{\tokT[0]} A (A + \fee x) \Big )(A + \fee x')
        \\ & = 
        \exchO{\tokT[1]} \fee \ammR[0] \ammR[1] \Big ( x' (A + \fee x) - x (A + \fee x') \Big )  
        \\ & - 
        \exchO{\tokT[0]} A \Big ( x' (A + \fee x')(A + \fee x) - x (A + \fee x)(A + \fee x') \Big )
        \\ & = 
        \exchO{\tokT[1]} \fee \ammR[0] \ammR[1] \Big ( x' (A + \fee x) - x (A + \fee x') \Big )  
        \\ & - 
        \exchO{\tokT[0]} A (x' - x) (A + \fee x')(A + \fee x)
        \\ & = 
        \exchO{\tokT[1]} \fee \ammR[0] \ammR[1] A(x' - x)  
        \\ & - 
        \exchO{\tokT[0]} A (x' - x) (A + \fee x')(A + \fee x)
        \\ & = 
        A (x' - x) \Big [ 
        \exchO{\tokT[1]} \fee \ammR[0] \ammR[1] - 
        \exchO{\tokT[0]} (A + \fee x')(A + \fee x)
        \Big ]
    \end{align*}

    Hence, 
    \[
        \Delta(x) = A^2 (x' - x) \Big [ 
            \exchO{\tokT[1]} \fee \ammR[0] \ammR[1] - 
            \exchO{\tokT[0]} (A + \fee x')(A + \fee x)
            \Big ]
    \]

    Now, recall the definition~\ref{eq:partial-xmax}. We can find a useful equality condition starting from this: 

    \begin{align*}
        & x' =
        \frac
            {-\sqrt{\exchO{\tokT[0]}} \ammR[0] - \sqrt{\exchO{\tokT[0]}} \fee x_0 + \sqrt{\exchO{\tokT[1]} \fee \ammR[0] \ammR[1]}}
            {\sqrt{\exchO{\tokT[0]}} \fee}
        \\ \iff & 
        \sqrt{\exchO{\tokT[0]}} \fee x' =
            -\sqrt{\exchO{\tokT[0]}} (\ammR[0] + \fee x_0) + \sqrt{\exchO{\tokT[1]} \fee \ammR[0] \ammR[1]}
        \\ \iff & 
        \sqrt{\exchO{\tokT[0]}} \fee x'  + \sqrt{\exchO{\tokT[0]}} A =
         \sqrt{\exchO{\tokT[1]} \fee \ammR[0] \ammR[1]}
        \\ \iff & 
        \sqrt{\exchO{\tokT[0]}} (A + \fee x') =
         \sqrt{\exchO{\tokT[1]} \fee \ammR[0] \ammR[1]}
        \\ \iff & 
        \exchO{\tokT[0]} (A + \fee x')^2 =
         \exchO{\tokT[1]} \fee \ammR[0] \ammR[1]
    \end{align*}
    
    Now recall the definition of $\Delta(x)$. With the above equality we get: 

    \begin{align*}
        \Delta(x) & = A^2 (x' - x) \Big [ 
            \exchO{\tokT[1]} \fee \ammR[0] \ammR[1] - 
            \exchO{\tokT[0]} (A + \fee x')(A + \fee x)
            \Big ]
        \\ & = 
        A^2 (x' - x) \Big [ 
            \exchO{\tokT[0]} (A + \fee x')^2 - 
            \exchO{\tokT[0]} (A + \fee x')(A + \fee x)
            \Big ]
        \\ & = 
        A^2 (x' - x) \Big [ 
            \exchO{\tokT[0]} (A + \fee x') ((A + \fee x') - (A + \fee x))
            \Big ]
        \\ & = 
        A^2 (x' - x) \Big [ 
            \exchO{\tokT[0]} (A + \fee x') \fee (x' - x)
            \Big ]
        \\ & = 
        A^2 \exchO{\tokT[0]} (A + \fee x') \fee (x' - x)^2
    \end{align*}

    Finally, combining this with \eqref{eq:D} yields: 
    \begin{align*}
        & f(x') > f(x)
        \\ \iff & 
        \Delta(x) > 0
        \\ \iff & 
        A^2 \exchO{\tokT[0]} (A + \fee x') \fee (x' - x)^2 > 0
        \\ \iff & 
        (x' - x)^2 > 0
        \\ \iff & 
        x' - x \neq 0
        \\ \iff & 
        x' \neq x
    \end{align*}

    \marconote{Devo dimostrare che la x' è positiva? Forse si, per questo serve la fee minore di 1}
    
    Since $A > 0$, $\exchO{\tokT[0]} > 0$, $\fee > 0$, $x' > 0$ and $x' \neq x$ by hypothesis. 
\end{proofof}

\begin{proofof}{Lemma}{lem:arbitrage:optimal-unique}
    
\end{proofof}