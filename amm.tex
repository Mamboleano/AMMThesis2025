\section{AMM model}

\mypar{Tokens}

We assume a set  $\TokU[0]$ of \keyterm{atomic token types},
which represent native cryptocurrencies and application-specific tokens.
For instance, $\TokU[0]$ may include ETH, the native cryptocurrency
of Ethereum, and WBTC, \ie
% \footnote{\url{https://wbtc.network/}}, 
Bitcoins wrapped with the ERC20 interface for Ethereum tokens.
A \keyterm{minted token type} is an unordered pair of distinct 
atomic token types:
if $\tokT[0]$ and $\tokT[1]$ are atomic token types 
and $\tokT[0] \neq \tokT[1]$,
then the minted token type $\tokM{\tokT[0]}{\tokT[1]}$ 
represents shares in an AMM holding reserves of $\tokT[0]$ and $\tokT[1]$.
We denote by $\TokU[1]$ the set of minted token types.
In our model, tokens are \emph{fungible},
\ie individual units of the same type are interchangeable.
This means that amounts of tokens of the same type
can be split into smaller parts,
and two amounts of tokens of the same type can be joined.
We use $\valV, \valVi, \ammR, \ammRi, x, x'$ to range over
nonnegative real numbers ($\RNN$).
We write $\TokU$ for the universe of all token types, 
\ie $\TokU = \TokU[0] \cup \TokU[1]$,
and we use $\tokT,\tokTi,\ldots$ to range over $\TokU$.
We write \mbox{$\ammR:\tokT$} to denote $\ammR$ units
of a token of type $\tokT$, either atomic or minted. 

\mypar{Users and Wallets}

We assume a set of \keyterm{users} $\PmvU$,
ranged over by $\pmvA, \pmvAi, \ldots$
We model the \keyterm{wallet} of a user $\pmvA$ as a term
$\wal{\pmvA}{\tokBal}$, where the finite partial map
$\tokBal \in \TokU \rightharpoonup \RNN$
represents $\pmvA$'s token balance.

\mypar{AMM states}

We model an \keyterm{AMM} holding reserves of \mbox{$\ammR[0]:\tokT[0]$} and
\mbox{$\ammR[1]:\tokT[1]$} (with $\tokT[0] \neq \tokT[1]$) 
as an unordered pair \mbox{$\amm{\ammR[0]:\tokT[0]}{\ammR[1]:\tokT[1]}$}.
Since the order of the token reserves in an AMM is immaterial, 
the terms \mbox{$\amm{\ammR[0]:\tokT[0]}{\ammR[1]:\tokT[1]}$}
and \mbox{$\amm{\ammR[1]:\tokT[1]}{\ammR[0]:\tokT[0]}$}
denote exactly the same AMM.
\bartnote{dire esplicitamente che sono entrambi $>0$}

\mypar{Blockchain states}

We model the interaction between users and AMMs
as a labelled transition system (LTS).
Its labels represent blockchain \keyterm{transactions}, while
the \keyterm{states} $\confG, \confGi, \confD, \ldots$ are 
finite non-empty compositions of wallets and AMMs.
Formally, states are terms of the form:
\[
\wal{\pmvA[1]}{\tokBal[1]} \mid \cdots \mid \wal{\pmvA[n]}{\tokBal[n]}
\mid
\amm{\ammR[1]:\tokT[1]}{\ammRi[1]:\tokTi[1]}
\mid \cdots \mid
\amm{\ammR[k]:\tokT[k]}{\ammRi[k]:\tokTi[k]}
\]
and subject to the following conditions. For all $i \neq j$:
\begin{enumerate}
\item $\pmvA[i] \neq \pmvA[j]$ 
  (each user has a single wallet);
\item
$\setenum{\tokT[i],\tokTi[i]} \neq \setenum{\tokT[j],\tokTi[j]}$ 
(distinct AMMs cannot hold exactly the same token types).
\end{enumerate}

Note that these conditions allow AMMs 
to have a common token type $\tokT$,
\eg as in \mbox{$\amm{\ammR[1]:\tokT[1]}{\ammR:\tokT}$},
\mbox{$\amm{\ammRi:\tokT}{\ammR[2]:\tokT[2]}$}, 
thus enabling indirect trades between token pairs
not directly provided by any AMM.
%
A state is \emph{initial}
when it has no AMMs, and its wallets hold only atomic tokens.
We stipulate that the ordering of terms in a state is immaterial.
Hence, we consider two states $\confG$ and $\confGi$ to be equivalent 
when they contain the same terms (regardless of their order). 
For a term $Q$ and a state $\confG$, we write $Q \in \confG$
when $\confG = Q \mid \confGi$, for some $\confGi$.

\mypar{Transactions}

State transitions are triggered by transactions
 $\ltsLabel, \ltsLabeli, \ldots$, which can have the following form
(where $\tokT[0]$ and $\tokT[1]$ are atomic tokens):
\[
\actAmmSwapExact{\pmvA}{}{\valV}{\tokT[0]}{\tokT[1]}
\]
With this transactions, 
$\pmvA$ tranfers $\valV:\tokT[0]$ to an AMM
$\amm{\ammR[0]:\tokT[0]}{\ammR[1]:\tokT[1]}$,
receiving in return some units of $\tokT[1]$, which are removed from the AMM.


We denote with
$\txType{\txT}$ the type of $\txT$
(\ie, $\ammDepositOp$, $\ammSwapOp$, or $\ammRedeemOp$),
with $\txWal{\txT}$ the user whose wallet is affected by $\txT$,
and with $\txTok{\txT}$ the set of token types affected by $\txT$.
For example, if
$\txT = \actAmmSwapExact{\pmvA}{}{\valV}{\tokT[0]}{\tokT[1]}$,
then
$\txType{\txT} = \ammSwapOp$,
$\txWal{\txT} = \pmvA$, and
$\txTok{\txT} = \setenum{\tokT[0],\tokT[1]}$.

\mypar{Token supply}

We use $\supply[\confG]{\tokT}$ to denote the \keyterm{supply} of a token type $\tokT$ in a state $\confG$, defined 
as the sum of the reserves of $\tokT$ in all the wallets and the AMMs
in $\confG$.
Formally, we define $\supply[\confG]{\tokT}$
by induction on the structure of states as follows:
\begin{equation*}
  \supply[\walA{\tokBal}]{\tokT} = 
  \begin{cases}
    \tokBal \tokT & \text{if $\tokT \in \dom{\tokBal}$} \\
    0 &\text{otherwise }
  \end{cases}
  \qquad
  \supply[\amm{\ammR[0]:\tokT[0]}{\ammR[1]:\tokT[1]}]{\tokT} =
  \begin{cases}
    \ammR[i] & \text{if $\tokT = \tokT[i]$} \\
    0 & \text{otherwise}
  \end{cases}
  \qquad
  \supply[\confG \mid \confGi]{\tokT} = \supply[\confG]{\tokT} + \supply[\confGi]{\tokT}
\end{equation*}

\mypar{AMM semantics: swap}

Any user $\pmvA$ can swap $\valV$ units of $\tokT[0]$ in her wallet
for some units of $\tokT[1]$ in an AMM 
\mbox{$\amm{\ammR[0]:\tokT[0]}{\ammR[1]:\tokT[1]}$}
through the transaction
\mbox{$\actAmmSwapExact{\pmvA}{}{\valV}{\tokT[0]}{\tokT[1]}$}.
Symmetrically, $\pmvA$ can swap $\valV$ of her units of $\tokT[1]$ 
for units of $\tokT[0]$ in the AMM through a transaction 
\mbox{$\actAmmSwapExact{\pmvA}{}{\valV}{\tokT[1]}{\tokT[0]}$}.
%
The \keyterm{swap rate} $\SX{x,\ammR[0],\ammR[1]}$
determines the amount of \emph{output tokens} $\tokT[1]$
that a user receives upon an amount of $x$ \emph{input tokens} $\tokT[0]$ 
in an AMM \mbox{$\amm{\ammR[0]:\tokT[0]}{\ammR[1]:\tokT[1]}$}.
\[
\irule
{
  \begin{array}{l}
    \tokBal{\tokT[0]} \geq x
    \qquad
    y = x \cdot \SX{x,\ammR[0],\ammR[1]} < \ammR[1]
  \end{array}
}
{\begin{array}{l}
   \walA{\tokBal}
   \mid
   \amm{\ammR[0]:\tokT[0]}{\ammR[1]:\tokT[1]}
   \mid
   \confG
   \xrightarrow{\actAmmSwapExact{\pmvA}{}{x}{\tokT[0]}{\tokT[1]}}
   \\[4pt]
   \walA{\tokBal - x:\tokT[0] + y:\tokT[1]}
   \mid
   \amm{\ammR[0]+ x:\tokT[0]}{\ammR[1]-y:\tokT[1]}
   \mid
   \confG
   \hspace{-5pt}
 \end{array}
}
\nrule{[Swap]}
\]

In this paper, we use \albnote{detta così sembra che tutto quello che segue è specifico per la constant product, ma sempre il caso? Explicitamente, lo `constant product' viene menzionato soltanto nel Lemma 2.11 e nel teorem 2.12. Poi ci sono tante lemmi/teoremi che usano la definizione specifica di SX come constant product. Una pulitina per rendere il tutto più chiaro e consistente basta :)} a specific swap rate function,
named \keyterm{constant product swap rate}.
This is used in mainstream AMM implementations,
like \eg in Uniswap v2~\cite{uniswapimpl}, 
Mooniswap~\cite{mooniswapimpl} and SushiSwap~\cite{sushiswapimpl}.

\albnote{Potrebbe essere utile indicare/menzionare le diff con il paper LMCS. }

\begin{defi}[Constant product swap rate]
  \label{def:const-prod}
  The constant product swap rate function is:  
  \[
    \SX{x,\ammR[0],\ammR[1]}
    \; = \;
    \frac{\fee \, \ammR[1]}{\ammR[0] + \fee \, x}
    \qquad\text{where } 
    \fee \in [0,1] 
  \]
\end{defi}

\albnote{$(0,1])$? Cioè, escludere 0?}

% \albnote{Altro: nota doppio pedice $\phi_\phi$ sopra.}
% \bartnote{risolto}

\bartnote{spiegare meglio}
This rules involves a \emph{trading fee} $1 - \fee$.
When the trading fee is zero (\ie, $\fee = 1$), 
the swap rate preserves the product between AMM reserves;
a higher fee, instead, results in reduced amounts of output tokens
received from swap actions.
Intuitively, the AMM retains a portion of the swapped amounts, 
but the overall reserves are still distributed among all minted tokens, 
thereby increasing the redeem rate of minted tokens.

\bartnote{rivedere} \albnote{$\phi$ sparita? :)}
\bartnote{è sparita perché questo commento l'ho copiato da LMCS. Ci va aggiunta}
The constant product swap rate ensures that,
if an AMM $\amm{\ammR[0]:\tokT[0]}{\ammR[1]:\tokT[1]}$ evolves into
$\amm{\ammR[0]+x:\tokT[0]}{\ammR[1]-y:\tokT[1]}$ upon a swap, 
then the product between the reserves is preserved:
\[
(\ammR[0]+x) (\ammR[1]-y)
\; = \;
(\ammR[0]+x) \Big( \ammR[1]-x \cdot \frac{\ammR[1]}{\ammR[0] + x} \Big)
\; = \;
\ammR[0] \ammR[1]
\]

\mypar{Net worth and gain}

% The incentive for users to participate in AMMs is to increase
The \keyterm{net worth} of a user $\pmvA$ is a measure of $\pmvA$'s 
wealth in tokens (both atomic and minted). 
Formally, we define the net worth of $\pmvA$ in a state $\confG$ as:
\begin{equation} 
  \label{eq:net-worth:user}
  W_{\pmvA}(\confG) \; = \; \begin{cases}
    \sum_{\tokT \in \dom{\tokBal}}
    \tokBal{\tokT} \cdot \exchO[\confG]{\tokT}
  & \text{if $\walA{\tokBal} \in \confG$} 
  \\[4pt]
  0 & \text{otherwise}
  \end{cases}
\end{equation}

We denote by $\gain[\confG]{\pmvA}{\bcB}$ the \keyterm{gain} of user $\pmvA$
upon performing a sequence of transactions $\bcB$ enabled in state $\confG$
(if $\bcB$ is not enabled in $\confG$, we stipulate that the gain is zero):
\begin{equation}
  \label{def:gain:A}
  \gain[\confG]{\pmvA}{\bcB}
  \; = \;
  W_{\pmvA}(\confGi) -W_{\pmvA}(\confG)
  \qquad \text{if $\confG \xrightarrow{\bcB} \confGi$}
\end{equation}
